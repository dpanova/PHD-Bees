%% Generated by Sphinx.
\def\sphinxdocclass{report}
\documentclass[letterpaper,10pt,english]{sphinxmanual}
\ifdefined\pdfpxdimen
   \let\sphinxpxdimen\pdfpxdimen\else\newdimen\sphinxpxdimen
\fi \sphinxpxdimen=.75bp\relax
\ifdefined\pdfimageresolution
    \pdfimageresolution= \numexpr \dimexpr1in\relax/\sphinxpxdimen\relax
\fi
%% let collapsible pdf bookmarks panel have high depth per default
\PassOptionsToPackage{bookmarksdepth=5}{hyperref}

\PassOptionsToPackage{booktabs}{sphinx}
\PassOptionsToPackage{colorrows}{sphinx}

\PassOptionsToPackage{warn}{textcomp}
\usepackage[utf8]{inputenc}
\ifdefined\DeclareUnicodeCharacter
% support both utf8 and utf8x syntaxes
  \ifdefined\DeclareUnicodeCharacterAsOptional
    \def\sphinxDUC#1{\DeclareUnicodeCharacter{"#1}}
  \else
    \let\sphinxDUC\DeclareUnicodeCharacter
  \fi
  \sphinxDUC{00A0}{\nobreakspace}
  \sphinxDUC{2500}{\sphinxunichar{2500}}
  \sphinxDUC{2502}{\sphinxunichar{2502}}
  \sphinxDUC{2514}{\sphinxunichar{2514}}
  \sphinxDUC{251C}{\sphinxunichar{251C}}
  \sphinxDUC{2572}{\textbackslash}
\fi
\usepackage{cmap}
\usepackage[T1]{fontenc}
\usepackage{amsmath,amssymb,amstext}
\usepackage{babel}



\usepackage{tgtermes}
\usepackage{tgheros}
\renewcommand{\ttdefault}{txtt}



\usepackage[Bjarne]{fncychap}
\usepackage{sphinx}

\fvset{fontsize=auto}
\usepackage{geometry}


% Include hyperref last.
\usepackage{hyperref}
% Fix anchor placement for figures with captions.
\usepackage{hypcap}% it must be loaded after hyperref.
% Set up styles of URL: it should be placed after hyperref.
\urlstyle{same}

\addto\captionsenglish{\renewcommand{\contentsname}{Contents:}}

\usepackage{sphinxmessages}
\setcounter{tocdepth}{1}



\title{Bee Transformer}
\date{May 06, 2024}
\release{}
\author{Denitsa Panova}
\newcommand{\sphinxlogo}{\vbox{}}
\renewcommand{\releasename}{}
\makeindex
\begin{document}

\ifdefined\shorthandoff
  \ifnum\catcode`\=\string=\active\shorthandoff{=}\fi
  \ifnum\catcode`\"=\active\shorthandoff{"}\fi
\fi

\pagestyle{empty}
\sphinxmaketitle
\pagestyle{plain}
\sphinxtableofcontents
\pagestyle{normal}
\phantomsection\label{\detokenize{index::doc}}


\sphinxstepscope


\chapter{Bee Transformer}
\label{\detokenize{modules:bee-transformer}}\label{\detokenize{modules::doc}}
\sphinxstepscope


\section{Bee Literature Review module}
\label{\detokenize{BeeLitReview:module-BeeLitReview}}\label{\detokenize{BeeLitReview:bee-literature-review-module}}\label{\detokenize{BeeLitReview::doc}}\index{module@\spxentry{module}!BeeLitReview@\spxentry{BeeLitReview}}\index{BeeLitReview@\spxentry{BeeLitReview}!module@\spxentry{module}}\index{BeeLitReview (class in BeeLitReview)@\spxentry{BeeLitReview}\spxextra{class in BeeLitReview}}

\begin{fulllineitems}
\phantomsection\label{\detokenize{BeeLitReview:BeeLitReview.BeeLitReview}}
\pysigstartsignatures
\pysiglinewithargsret{\sphinxbfcode{\sphinxupquote{class\DUrole{w}{ }}}\sphinxcode{\sphinxupquote{BeeLitReview.}}\sphinxbfcode{\sphinxupquote{BeeLitReview}}}{\sphinxparam{\DUrole{n}{logname}\DUrole{o}{=}\DUrole{default_value}{\textquotesingle{}BeeLitReview.log\textquotesingle{}}}\sphinxparamcomma \sphinxparam{\DUrole{n}{already\_scraped}\DUrole{o}{=}\DUrole{default_value}{True}}\sphinxparamcomma \sphinxparam{\DUrole{n}{query}\DUrole{o}{=}\DUrole{default_value}{\textquotesingle{}bee acoustic machine learning\textquotesingle{}}}\sphinxparamcomma \sphinxparam{\DUrole{n}{scraped\_file\_name}\DUrole{o}{=}\DUrole{default_value}{\textquotesingle{}scraped\_articles.csv\textquotesingle{}}}\sphinxparamcomma \sphinxparam{\DUrole{n}{scraped\_file\_validation}\DUrole{o}{=}\DUrole{default_value}{\textquotesingle{}scraped\_articles\_validation.csv\textquotesingle{}}}\sphinxparamcomma \sphinxparam{\DUrole{n}{to\_review\_file\_name}\DUrole{o}{=}\DUrole{default_value}{\textquotesingle{}to\_review.xlsx\textquotesingle{}}}}{}
\pysigstopsignatures
\sphinxAtStartPar
Bases: \sphinxcode{\sphinxupquote{object}}

\sphinxAtStartPar
Class to create an automated pdf report to help facilitate depth and breadth of literature research.
\begin{quote}\begin{description}
\sphinxlineitem{Parameters}\begin{itemize}
\item {} 
\sphinxAtStartPar
\sphinxstyleliteralstrong{\sphinxupquote{logname}} (\sphinxstyleliteralemphasis{\sphinxupquote{str}}) \textendash{} name of the log to save the results

\item {} 
\sphinxAtStartPar
\sphinxstyleliteralstrong{\sphinxupquote{already\_scraped}} (\sphinxstyleliteralemphasis{\sphinxupquote{bool}}) \textendash{} whether the data needs to be scraped or already exists locally

\item {} 
\sphinxAtStartPar
\sphinxstyleliteralstrong{\sphinxupquote{query}} (\sphinxstyleliteralemphasis{\sphinxupquote{str}}) \textendash{} query to be scraped from the website

\item {} 
\sphinxAtStartPar
\sphinxstyleliteralstrong{\sphinxupquote{scraped\_file\_name}} (\sphinxstyleliteralemphasis{\sphinxupquote{str}}) \textendash{} name of the file to store the scraped data or where the data exists

\item {} 
\sphinxAtStartPar
\sphinxstyleliteralstrong{\sphinxupquote{scraped\_file\_validation}} (\sphinxstyleliteralemphasis{\sphinxupquote{str}}) \textendash{} name of the file to validate the existent data in terms of format

\item {} 
\sphinxAtStartPar
\sphinxstyleliteralstrong{\sphinxupquote{to\_review\_file\_name}} (\sphinxstyleliteralemphasis{\sphinxupquote{excel}}) \textendash{} name of the file to validate TSP results

\end{itemize}

\sphinxlineitem{Returns}
\sphinxAtStartPar
BeeLitReview object

\sphinxlineitem{Return type}
\sphinxAtStartPar
{\hyperref[\detokenize{BeeLitReview:BeeLitReview.BeeLitReview}]{\sphinxcrossref{BeeLitReview}}}

\end{description}\end{quote}
\index{article\_scrape() (BeeLitReview.BeeLitReview method)@\spxentry{article\_scrape()}\spxextra{BeeLitReview.BeeLitReview method}}

\begin{fulllineitems}
\phantomsection\label{\detokenize{BeeLitReview:BeeLitReview.BeeLitReview.article_scrape}}
\pysigstartsignatures
\pysiglinewithargsret{\sphinxbfcode{\sphinxupquote{article\_scrape}}}{\sphinxparam{\DUrole{n}{article}}}{}
\pysigstopsignatures
\sphinxAtStartPar
Function to scrape the important aspects from an article result \sphinxhyphen{} title, category, date, reads, citations, authors, url and abstract
\begin{quote}\begin{description}
\sphinxlineitem{Parameters}
\sphinxAtStartPar
\sphinxstyleliteralstrong{\sphinxupquote{article}} (\sphinxstyleliteralemphasis{\sphinxupquote{undetected\_chromedriver.webelement.WebElement}}) \textendash{} article result

\sphinxlineitem{Returns}
\sphinxAtStartPar
scraped date in the df object

\sphinxlineitem{Return type}
\sphinxAtStartPar
dataframe

\end{description}\end{quote}

\end{fulllineitems}

\index{calculate\_TSP() (BeeLitReview.BeeLitReview method)@\spxentry{calculate\_TSP()}\spxextra{BeeLitReview.BeeLitReview method}}

\begin{fulllineitems}
\phantomsection\label{\detokenize{BeeLitReview:BeeLitReview.BeeLitReview.calculate_TSP}}
\pysigstartsignatures
\pysiglinewithargsret{\sphinxbfcode{\sphinxupquote{calculate\_TSP}}}{\sphinxparam{\DUrole{n}{num\_articles}\DUrole{o}{=}\DUrole{default_value}{50}}\sphinxparamcomma \sphinxparam{\DUrole{n}{path\_file\_name}\DUrole{o}{=}\DUrole{default_value}{\textquotesingle{}path.csv\textquotesingle{}}}}{}
\pysigstopsignatures
\sphinxAtStartPar
Calculates the Travel Salesmen Path based on the similarity between the abstracts.
\begin{quote}\begin{description}
\sphinxlineitem{Parameters}\begin{itemize}
\item {} 
\sphinxAtStartPar
\sphinxstyleliteralstrong{\sphinxupquote{num\_articles}} (\sphinxstyleliteralemphasis{\sphinxupquote{int}}) \textendash{} number of articles to bee looked into

\item {} 
\sphinxAtStartPar
\sphinxstyleliteralstrong{\sphinxupquote{path\_file\_name}} (\sphinxstyleliteralemphasis{\sphinxupquote{str}}) \textendash{} name of the file to store the path

\end{itemize}

\sphinxlineitem{Returns}
\sphinxAtStartPar
Returns dataframe with the first num\_articles and saves to data to to\_review\_file\_name.csv for further inspection

\end{description}\end{quote}

\end{fulllineitems}

\index{calculate\_similarity() (BeeLitReview.BeeLitReview method)@\spxentry{calculate\_similarity()}\spxextra{BeeLitReview.BeeLitReview method}}

\begin{fulllineitems}
\phantomsection\label{\detokenize{BeeLitReview:BeeLitReview.BeeLitReview.calculate_similarity}}
\pysigstartsignatures
\pysiglinewithargsret{\sphinxbfcode{\sphinxupquote{calculate\_similarity}}}{}{}
\pysigstopsignatures
\sphinxAtStartPar
Calculate the pairwise cosine similarity between each abstract based on the already calculated embeddings
\begin{quote}\begin{description}
\sphinxlineitem{Returns}
\sphinxAtStartPar
similarity dataframe with columns pair0, pair1 and cos. Saves the data to similarity\_df.csv

\end{description}\end{quote}

\end{fulllineitems}

\index{clustering\_evaluation() (BeeLitReview.BeeLitReview method)@\spxentry{clustering\_evaluation()}\spxextra{BeeLitReview.BeeLitReview method}}

\begin{fulllineitems}
\phantomsection\label{\detokenize{BeeLitReview:BeeLitReview.BeeLitReview.clustering_evaluation}}
\pysigstartsignatures
\pysiglinewithargsret{\sphinxbfcode{\sphinxupquote{clustering\_evaluation}}}{\sphinxparam{\DUrole{n}{cosine\_threshold}\DUrole{o}{=}\DUrole{default_value}{0.3}}\sphinxparamcomma \sphinxparam{\DUrole{n}{year}\DUrole{o}{=}\DUrole{default_value}{2023}}\sphinxparamcomma \sphinxparam{\DUrole{n}{type}\DUrole{o}{=}\DUrole{default_value}{{[}\textquotesingle{}Article\textquotesingle{}, \textquotesingle{}Conference Paper\textquotesingle{}, \textquotesingle{}Preprint\textquotesingle{}, \textquotesingle{}Patent\textquotesingle{}, \textquotesingle{}Thesis\textquotesingle{}{]}}}\sphinxparamcomma \sphinxparam{\DUrole{n}{read}\DUrole{o}{=}\DUrole{default_value}{10}}\sphinxparamcomma \sphinxparam{\DUrole{n}{citation}\DUrole{o}{=}\DUrole{default_value}{1}}}{}
\pysigstopsignatures
\end{fulllineitems}

\index{encode\_with\_transformers() (BeeLitReview.BeeLitReview method)@\spxentry{encode\_with\_transformers()}\spxextra{BeeLitReview.BeeLitReview method}}

\begin{fulllineitems}
\phantomsection\label{\detokenize{BeeLitReview:BeeLitReview.BeeLitReview.encode_with_transformers}}
\pysigstartsignatures
\pysiglinewithargsret{\sphinxbfcode{\sphinxupquote{encode\_with\_transformers}}}{\sphinxparam{\DUrole{n}{model\_id}\DUrole{o}{=}\DUrole{default_value}{\textquotesingle{}sentence\sphinxhyphen{}transformers/all\sphinxhyphen{}mpnet\sphinxhyphen{}base\sphinxhyphen{}v2\textquotesingle{}}}\sphinxparamcomma \sphinxparam{\DUrole{n}{abstract\_col}\DUrole{o}{=}\DUrole{default_value}{\textquotesingle{}Abstract\textquotesingle{}}}}{}
\pysigstopsignatures
\sphinxAtStartPar
Function to encode the abstract text with HuggingFace transformers. Embeddings are saved to embeddings
\begin{quote}\begin{description}
\sphinxlineitem{Parameters}\begin{itemize}
\item {} 
\sphinxAtStartPar
\sphinxstyleliteralstrong{\sphinxupquote{model\_id}} (\sphinxstyleliteralemphasis{\sphinxupquote{str}}) \textendash{} HuggingFace transformers’ model

\item {} 
\sphinxAtStartPar
\sphinxstyleliteralstrong{\sphinxupquote{abstract\_col}} (\sphinxstyleliteralemphasis{\sphinxupquote{str}}) \textendash{} Column indicating the abstract

\end{itemize}

\sphinxlineitem{Returns}
\sphinxAtStartPar
embeddings in a list format

\end{description}\end{quote}

\end{fulllineitems}

\index{pdf\_graph() (BeeLitReview.BeeLitReview method)@\spxentry{pdf\_graph()}\spxextra{BeeLitReview.BeeLitReview method}}

\begin{fulllineitems}
\phantomsection\label{\detokenize{BeeLitReview:BeeLitReview.BeeLitReview.pdf_graph}}
\pysigstartsignatures
\pysiglinewithargsret{\sphinxbfcode{\sphinxupquote{pdf\_graph}}}{\sphinxparam{\DUrole{n}{pdf}}\sphinxparamcomma \sphinxparam{\DUrole{n}{x}}\sphinxparamcomma \sphinxparam{\DUrole{n}{y}}\sphinxparamcomma \sphinxparam{\DUrole{n}{w}}\sphinxparamcomma \sphinxparam{\DUrole{n}{h}}\sphinxparamcomma \sphinxparam{\DUrole{n}{with\_code}\DUrole{o}{=}\DUrole{default_value}{True}}\sphinxparamcomma \sphinxparam{\DUrole{n}{plot\_code}\DUrole{o}{=}\DUrole{default_value}{\textquotesingle{}\textquotesingle{}}}\sphinxparamcomma \sphinxparam{\DUrole{n}{filename}\DUrole{o}{=}\DUrole{default_value}{\textquotesingle{}\textquotesingle{}}}}{}
\pysigstopsignatures
\sphinxAtStartPar
Generates in the pdf file
\begin{quote}\begin{description}
\sphinxlineitem{Parameters}\begin{itemize}
\item {} 
\sphinxAtStartPar
\sphinxstyleliteralstrong{\sphinxupquote{pdf}} (\sphinxstyleliteralemphasis{\sphinxupquote{FPDF}}) \textendash{} FPDF instance

\item {} 
\sphinxAtStartPar
\sphinxstyleliteralstrong{\sphinxupquote{x}} (\sphinxstyleliteralemphasis{\sphinxupquote{int}}) \textendash{} x\sphinxhyphen{}coordinate of the plot

\item {} 
\sphinxAtStartPar
\sphinxstyleliteralstrong{\sphinxupquote{y}} (\sphinxstyleliteralemphasis{\sphinxupquote{int}}) \textendash{} y\sphinxhyphen{}coordinate of the plot

\item {} 
\sphinxAtStartPar
\sphinxstyleliteralstrong{\sphinxupquote{w}} (\sphinxstyleliteralemphasis{\sphinxupquote{int0}}) \textendash{} w\sphinxhyphen{}coordinate of the plot

\item {} 
\sphinxAtStartPar
\sphinxstyleliteralstrong{\sphinxupquote{h}} (\sphinxstyleliteralemphasis{\sphinxupquote{int}}) \textendash{} h\sphinxhyphen{}coordinate of the plot

\item {} 
\sphinxAtStartPar
\sphinxstyleliteralstrong{\sphinxupquote{with\_code}} (\sphinxstyleliteralemphasis{\sphinxupquote{bool}}) \textendash{} if code will be executed for teh plot to be created or the plot is saved as a picture

\item {} 
\sphinxAtStartPar
\sphinxstyleliteralstrong{\sphinxupquote{plot\_code}} (\sphinxstyleliteralemphasis{\sphinxupquote{str}}) \textendash{} code to be executed for the plot to be created

\item {} 
\sphinxAtStartPar
\sphinxstyleliteralstrong{\sphinxupquote{filename}} (\sphinxstyleliteralemphasis{\sphinxupquote{str}}) \textendash{} name of the saved picture

\end{itemize}

\sphinxlineitem{Returns}
\sphinxAtStartPar
a graph in the pdf file

\end{description}\end{quote}

\end{fulllineitems}

\index{pdf\_report\_generate() (BeeLitReview.BeeLitReview method)@\spxentry{pdf\_report\_generate()}\spxextra{BeeLitReview.BeeLitReview method}}

\begin{fulllineitems}
\phantomsection\label{\detokenize{BeeLitReview:BeeLitReview.BeeLitReview.pdf_report_generate}}
\pysigstartsignatures
\pysiglinewithargsret{\sphinxbfcode{\sphinxupquote{pdf\_report\_generate}}}{\sphinxparam{\DUrole{n}{report\_file\_name}\DUrole{o}{=}\DUrole{default_value}{\textquotesingle{}lit\_report.pdf\textquotesingle{}}}}{}
\pysigstopsignatures
\sphinxAtStartPar
Generate automated pdf report based on the results from the class

\sphinxAtStartPar
The report has:
INTRO \sphinxhyphen{} Report title, author, data of the report and disclaimer and dependant variable distribution
DATA OVERVIEW \sphinxhyphen{} Time, word, result type, language distributions
Optimal ML Path
\sphinxhyphen{} Travelling Salesman Problem Solution
\sphinxhyphen{} Similarity Difference
\sphinxhyphen{} Time Saved
\sphinxhyphen{} TSP Results of Interest and graph visual representation
\sphinxhyphen{} Clustering Results
Suggested Reads
\begin{quote}\begin{description}
\sphinxlineitem{Parameters}
\sphinxAtStartPar
\sphinxstyleliteralstrong{\sphinxupquote{report\_file\_name}} (\sphinxstyleliteralemphasis{\sphinxupquote{str}}) \textendash{} name of the report file

\sphinxlineitem{Returns}
\sphinxAtStartPar
pdf file

\sphinxlineitem{Return type}
\sphinxAtStartPar
pdf

\end{description}\end{quote}

\end{fulllineitems}

\index{researchgate\_scraper() (BeeLitReview.BeeLitReview method)@\spxentry{researchgate\_scraper()}\spxextra{BeeLitReview.BeeLitReview method}}

\begin{fulllineitems}
\phantomsection\label{\detokenize{BeeLitReview:BeeLitReview.BeeLitReview.researchgate_scraper}}
\pysigstartsignatures
\pysiglinewithargsret{\sphinxbfcode{\sphinxupquote{researchgate\_scraper}}}{\sphinxparam{\DUrole{n}{username}\DUrole{o}{=}\DUrole{default_value}{\textquotesingle{}\textquotesingle{}}}\sphinxparamcomma \sphinxparam{\DUrole{n}{password}\DUrole{o}{=}\DUrole{default_value}{\textquotesingle{}\textquotesingle{}}}\sphinxparamcomma \sphinxparam{\DUrole{n}{url}\DUrole{o}{=}\DUrole{default_value}{\textquotesingle{}https://www.researchgate.net/login?\_sg=ITbsht6C\sphinxhyphen{}ko8ZSF49e9deV1BNgFqKApIltdEguj\_4uiZ9K\_WzI6gYtnTL5xsgogphkn5Z2RJTNuYqd9fMiGJfg\textquotesingle{}}}}{}
\pysigstopsignatures
\sphinxAtStartPar
Function to scrape the articles from ResearchGate
\begin{quote}\begin{description}
\sphinxlineitem{Parameters}\begin{itemize}
\item {} 
\sphinxAtStartPar
\sphinxstyleliteralstrong{\sphinxupquote{username}} (\sphinxstyleliteralemphasis{\sphinxupquote{str}}) \textendash{} username in the website

\item {} 
\sphinxAtStartPar
\sphinxstyleliteralstrong{\sphinxupquote{password}} (\sphinxstyleliteralemphasis{\sphinxupquote{str}}) \textendash{} password in the website

\item {} 
\sphinxAtStartPar
\sphinxstyleliteralstrong{\sphinxupquote{url}} (\sphinxstyleliteralemphasis{\sphinxupquote{str}}) \textendash{} url of the website

\end{itemize}

\sphinxlineitem{Returns}
\sphinxAtStartPar
scraped date in the df object and saved csv

\sphinxlineitem{Return type}
\sphinxAtStartPar
dataframe

\end{description}\end{quote}

\end{fulllineitems}

\index{scraped\_data\_enhancement() (BeeLitReview.BeeLitReview method)@\spxentry{scraped\_data\_enhancement()}\spxextra{BeeLitReview.BeeLitReview method}}

\begin{fulllineitems}
\phantomsection\label{\detokenize{BeeLitReview:BeeLitReview.BeeLitReview.scraped_data_enhancement}}
\pysigstartsignatures
\pysiglinewithargsret{\sphinxbfcode{\sphinxupquote{scraped\_data\_enhancement}}}{\sphinxparam{\DUrole{n}{stats\_col}\DUrole{o}{=}\DUrole{default_value}{\textquotesingle{}Right Category\textquotesingle{}}}\sphinxparamcomma \sphinxparam{\DUrole{n}{cat\_col}\DUrole{o}{=}\DUrole{default_value}{\textquotesingle{}Left Category\textquotesingle{}}}\sphinxparamcomma \sphinxparam{\DUrole{n}{key\_col}\DUrole{o}{=}\DUrole{default_value}{\textquotesingle{}Title\textquotesingle{}}}\sphinxparamcomma \sphinxparam{\DUrole{n}{abstract\_col}\DUrole{o}{=}\DUrole{default_value}{\textquotesingle{}Abstract\textquotesingle{}}}\sphinxparamcomma \sphinxparam{\DUrole{n}{date\_col}\DUrole{o}{=}\DUrole{default_value}{\textquotesingle{}Date\textquotesingle{}}}}{}
\pysigstopsignatures
\sphinxAtStartPar
Function to extract dates, result type \sphinxhyphen{} article/ etc, tokenize the abstract, split the important stats such as citations and keep only the English text.
\begin{quote}\begin{description}
\sphinxlineitem{Parameters}\begin{itemize}
\item {} 
\sphinxAtStartPar
\sphinxstyleliteralstrong{\sphinxupquote{stats\_col}} (\sphinxstyleliteralemphasis{\sphinxupquote{str}}) \textendash{} Column which has the stats data

\item {} 
\sphinxAtStartPar
\sphinxstyleliteralstrong{\sphinxupquote{cat\_col}} (\sphinxstyleliteralemphasis{\sphinxupquote{str}}) \textendash{} Column which has the result type

\item {} 
\sphinxAtStartPar
\sphinxstyleliteralstrong{\sphinxupquote{key\_col}} (\sphinxstyleliteralemphasis{\sphinxupquote{str}}) \textendash{} Column which will be used for the duplication reduction (last entry is kept in the data)

\item {} 
\sphinxAtStartPar
\sphinxstyleliteralstrong{\sphinxupquote{abstract\_col}} (\sphinxstyleliteralemphasis{\sphinxupquote{str}}) \textendash{} Column which has the abstract of the text

\item {} 
\sphinxAtStartPar
\sphinxstyleliteralstrong{\sphinxupquote{date\_col}} (\sphinxstyleliteralemphasis{\sphinxupquote{str}}) \textendash{} Colum which has the date information

\end{itemize}

\sphinxlineitem{Returns}
\sphinxAtStartPar
Returns pandas with new columns with the extracted information, refer to the df in the object. Additionally, the data is saved locally to \_enhanced file

\end{description}\end{quote}

\end{fulllineitems}

\index{validate\_scraped\_csv() (BeeLitReview.BeeLitReview method)@\spxentry{validate\_scraped\_csv()}\spxextra{BeeLitReview.BeeLitReview method}}

\begin{fulllineitems}
\phantomsection\label{\detokenize{BeeLitReview:BeeLitReview.BeeLitReview.validate_scraped_csv}}
\pysigstartsignatures
\pysiglinewithargsret{\sphinxbfcode{\sphinxupquote{validate\_scraped\_csv}}}{}{}
\pysigstopsignatures
\sphinxAtStartPar
”
Validate scraped data
\begin{quote}\begin{description}
\sphinxlineitem{Returns}
\sphinxAtStartPar
error if the data is not as expected

\end{description}\end{quote}

\end{fulllineitems}


\end{fulllineitems}


\sphinxstepscope


\section{Bee Data module}
\label{\detokenize{BeeData:module-BeeData}}\label{\detokenize{BeeData:bee-data-module}}\label{\detokenize{BeeData::doc}}\index{module@\spxentry{module}!BeeData@\spxentry{BeeData}}\index{BeeData@\spxentry{BeeData}!module@\spxentry{module}}\index{BeeData (class in BeeData)@\spxentry{BeeData}\spxextra{class in BeeData}}

\begin{fulllineitems}
\phantomsection\label{\detokenize{BeeData:BeeData.BeeData}}
\pysigstartsignatures
\pysiglinewithargsret{\sphinxbfcode{\sphinxupquote{class\DUrole{w}{ }}}\sphinxcode{\sphinxupquote{BeeData.}}\sphinxbfcode{\sphinxupquote{BeeData}}}{\sphinxparam{\DUrole{n}{acoustic\_folder}\DUrole{o}{=}\DUrole{default_value}{\textquotesingle{}data/SplitData/\textquotesingle{}}}\sphinxparamcomma \sphinxparam{\DUrole{n}{logname}\DUrole{o}{=}\DUrole{default_value}{\textquotesingle{}BeeData.log\textquotesingle{}}}\sphinxparamcomma \sphinxparam{\DUrole{n}{bee\_col}\DUrole{o}{=}\DUrole{default_value}{\textquotesingle{}label\textquotesingle{}}}\sphinxparamcomma \sphinxparam{\DUrole{n}{file\_name}\DUrole{o}{=}\DUrole{default_value}{\textquotesingle{}beeAnnotations\_enhanced.csv\textquotesingle{}}}\sphinxparamcomma \sphinxparam{\DUrole{n}{annotation\_file}\DUrole{o}{=}\DUrole{default_value}{\textquotesingle{}beeAnnotations.mlf\textquotesingle{}}}\sphinxparamcomma \sphinxparam{\DUrole{n}{start\_col\_name}\DUrole{o}{=}\DUrole{default_value}{\textquotesingle{}start\textquotesingle{}}}\sphinxparamcomma \sphinxparam{\DUrole{n}{end\_col\_name}\DUrole{o}{=}\DUrole{default_value}{\textquotesingle{}end\textquotesingle{}}}\sphinxparamcomma \sphinxparam{\DUrole{n}{file\_col\_name}\DUrole{o}{=}\DUrole{default_value}{\textquotesingle{}file name\textquotesingle{}}}\sphinxparamcomma \sphinxparam{\DUrole{n}{duration\_col\_name}\DUrole{o}{=}\DUrole{default_value}{\textquotesingle{}duration\textquotesingle{}}}\sphinxparamcomma \sphinxparam{\DUrole{n}{key\_col\_name}\DUrole{o}{=}\DUrole{default_value}{\textquotesingle{}index\textquotesingle{}}}}{}
\pysigstopsignatures
\sphinxAtStartPar
Bases: \sphinxcode{\sphinxupquote{object}}

\sphinxAtStartPar
Class to conduct data manipulation on audio files.
\begin{quote}\begin{description}
\sphinxlineitem{Parameters}\begin{itemize}
\item {} 
\sphinxAtStartPar
\sphinxstyleliteralstrong{\sphinxupquote{annotation\_file}} (\sphinxstyleliteralemphasis{\sphinxupquote{str}}) \textendash{} name of the .mlf file to be read

\item {} 
\sphinxAtStartPar
\sphinxstyleliteralstrong{\sphinxupquote{start\_col\_name}} (\sphinxstyleliteralemphasis{\sphinxupquote{str}}) \textendash{} name of the column with the start time

\item {} 
\sphinxAtStartPar
\sphinxstyleliteralstrong{\sphinxupquote{end\_col\_name}} (\sphinxstyleliteralemphasis{\sphinxupquote{str}}) \textendash{} name of the column with the end time

\item {} 
\sphinxAtStartPar
\sphinxstyleliteralstrong{\sphinxupquote{label\_col\_name}} (\sphinxstyleliteralemphasis{\sphinxupquote{str}}) \textendash{} name of the column with the bee/nobee label

\item {} 
\sphinxAtStartPar
\sphinxstyleliteralstrong{\sphinxupquote{file\_col\_name}} (\sphinxstyleliteralemphasis{\sphinxupquote{str}}) \textendash{} name of the column with the file name

\item {} 
\sphinxAtStartPar
\sphinxstyleliteralstrong{\sphinxupquote{duration\_col\_name}} (\sphinxstyleliteralemphasis{\sphinxupquote{str}}) \textendash{} column name for the duration

\end{itemize}

\end{description}\end{quote}
\index{annotation\_data\_creation() (BeeData.BeeData method)@\spxentry{annotation\_data\_creation()}\spxextra{BeeData.BeeData method}}

\begin{fulllineitems}
\phantomsection\label{\detokenize{BeeData:BeeData.BeeData.annotation_data_creation}}
\pysigstartsignatures
\pysiglinewithargsret{\sphinxbfcode{\sphinxupquote{annotation\_data\_creation}}}{\sphinxparam{\DUrole{n}{special\_action\_column}\DUrole{o}{=}\DUrole{default_value}{True}}\sphinxparamcomma \sphinxparam{\DUrole{n}{dict\_actions}\DUrole{o}{=}\DUrole{default_value}{\{\textquotesingle{}active day\textquotesingle{}: {[}\textquotesingle{}active \sphinxhyphen{} day\textquotesingle{}{]}, \textquotesingle{}missing queen\textquotesingle{}: {[}\textquotesingle{}missing queen\textquotesingle{}, \textquotesingle{}no\_queen\textquotesingle{}{]}, \textquotesingle{}swarming\textquotesingle{}: {[}\textquotesingle{}swarming\textquotesingle{}{]}\}}}}{}
\pysigstopsignatures
\sphinxAtStartPar
Function to create the dataset for annotation and save it locally under the specified file name.
\begin{quote}\begin{description}
\sphinxlineitem{Parameters}\begin{itemize}
\item {} 
\sphinxAtStartPar
\sphinxstyleliteralstrong{\sphinxupquote{special\_action\_column}} (\sphinxstyleliteralemphasis{\sphinxupquote{bool}}) \textendash{} boolean which indicates if the special action column will be calculated

\item {} 
\sphinxAtStartPar
\sphinxstyleliteralstrong{\sphinxupquote{dict\_actions}} (\sphinxstyleliteralemphasis{\sphinxupquote{dict}}) \textendash{} dictionary with the new column names as the key and the string match to be searched.Note that each string should be in a list

\end{itemize}

\sphinxlineitem{Returns}
\sphinxAtStartPar
pandas.DataFrame containing the filename, start time, end time, duration between both, bee/nobee label and action columns. The data is stores in the object.

\sphinxlineitem{Return type}
\sphinxAtStartPar
pandas.DataFrame

\end{description}\end{quote}

\end{fulllineitems}

\index{calc\_duration() (BeeData.BeeData method)@\spxentry{calc\_duration()}\spxextra{BeeData.BeeData method}}

\begin{fulllineitems}
\phantomsection\label{\detokenize{BeeData:BeeData.BeeData.calc_duration}}
\pysigstartsignatures
\pysiglinewithargsret{\sphinxbfcode{\sphinxupquote{calc\_duration}}}{\sphinxparam{\DUrole{n}{df}}}{}
\pysigstopsignatures
\sphinxAtStartPar
Calculates the duration between two time periods
\begin{quote}\begin{description}
\sphinxlineitem{Parameters}
\sphinxAtStartPar
\sphinxstyleliteralstrong{\sphinxupquote{df}} (\sphinxstyleliteralemphasis{\sphinxupquote{pandas.DataFrame}}) \textendash{} The dataframe to calculate duration in

\sphinxlineitem{Returns}
\sphinxAtStartPar
pd.Series containing the duration

\sphinxlineitem{Return type}
\sphinxAtStartPar
pandas.Series

\end{description}\end{quote}

\end{fulllineitems}

\index{create\_actions() (BeeData.BeeData method)@\spxentry{create\_actions()}\spxextra{BeeData.BeeData method}}

\begin{fulllineitems}
\phantomsection\label{\detokenize{BeeData:BeeData.BeeData.create_actions}}
\pysigstartsignatures
\pysiglinewithargsret{\sphinxbfcode{\sphinxupquote{create\_actions}}}{\sphinxparam{\DUrole{n}{df}}\sphinxparamcomma \sphinxparam{\DUrole{n}{dict\_actions}}}{}
\pysigstopsignatures
\sphinxAtStartPar
Creates new columns with labels based on  the provided dictionary
\begin{quote}\begin{description}
\sphinxlineitem{Parameters}\begin{itemize}
\item {} 
\sphinxAtStartPar
\sphinxstyleliteralstrong{\sphinxupquote{df}} (\sphinxstyleliteralemphasis{\sphinxupquote{pandas.DataFrame}}) \textendash{} The dataframe to calculate duration in

\item {} 
\sphinxAtStartPar
\sphinxstyleliteralstrong{\sphinxupquote{dict\_actions}} (\sphinxstyleliteralemphasis{\sphinxupquote{dict}}) \textendash{} dictionary with the new column names as the key and the string match to be searched.Note that each string should be in a list

\end{itemize}

\sphinxlineitem{Returns}
\sphinxAtStartPar
data frame with the new columns

\sphinxlineitem{Return type}
\sphinxAtStartPar
pandas.DataFrame

\end{description}\end{quote}

\end{fulllineitems}

\index{create\_validate\_data() (BeeData.BeeData method)@\spxentry{create\_validate\_data()}\spxextra{BeeData.BeeData method}}

\begin{fulllineitems}
\phantomsection\label{\detokenize{BeeData:BeeData.BeeData.create_validate_data}}
\pysigstartsignatures
\pysiglinewithargsret{\sphinxbfcode{\sphinxupquote{create\_validate\_data}}}{\sphinxparam{\DUrole{n}{sliced}\DUrole{o}{=}\DUrole{default_value}{True}}\sphinxparamcomma \sphinxparam{\DUrole{n}{file\_name}\DUrole{o}{=}\DUrole{default_value}{\textquotesingle{}annotation\_data\_type.csv\textquotesingle{}}}}{}
\pysigstopsignatures
\sphinxAtStartPar
Creates validation data set for the data types.
\begin{quote}\begin{description}
\sphinxlineitem{Parameters}\begin{itemize}
\item {} 
\sphinxAtStartPar
\sphinxstyleliteralstrong{\sphinxupquote{sliced}} (\sphinxstyleliteralemphasis{\sphinxupquote{bool}}) \textendash{} boolean to indicate whether to create validation based on the sliced data or not

\item {} 
\sphinxAtStartPar
\sphinxstyleliteralstrong{\sphinxupquote{file\_name}} (\sphinxstyleliteralemphasis{\sphinxupquote{basestring}}) \textendash{} name of the file to be saved to

\end{itemize}

\sphinxlineitem{Returns}
\sphinxAtStartPar
csv file with the validation data

\end{description}\end{quote}

\end{fulllineitems}

\index{data\_quality() (BeeData.BeeData method)@\spxentry{data\_quality()}\spxextra{BeeData.BeeData method}}

\begin{fulllineitems}
\phantomsection\label{\detokenize{BeeData:BeeData.BeeData.data_quality}}
\pysigstartsignatures
\pysiglinewithargsret{\sphinxbfcode{\sphinxupquote{data\_quality}}}{\sphinxparam{\DUrole{n}{nobee}\DUrole{o}{=}\DUrole{default_value}{False}}\sphinxparamcomma \sphinxparam{\DUrole{n}{path}\DUrole{o}{=}\DUrole{default_value}{\textquotesingle{}data\textquotesingle{}}}\sphinxparamcomma \sphinxparam{\DUrole{n}{min\_duration}\DUrole{o}{=}\DUrole{default_value}{2.0}}}{}
\pysigstopsignatures
\sphinxAtStartPar
Ensures quality data
\begin{quote}\begin{description}
\sphinxlineitem{Parameters}\begin{itemize}
\item {} 
\sphinxAtStartPar
\sphinxstyleliteralstrong{\sphinxupquote{path}} (\sphinxstyleliteralemphasis{\sphinxupquote{str}}) \textendash{} directory path for the wav and mp3 files

\item {} 
\sphinxAtStartPar
\sphinxstyleliteralstrong{\sphinxupquote{min\_duration}} (\sphinxstyleliteralemphasis{\sphinxupquote{float}}) \textendash{} minimum duration of the file segment in order to make sense to work with it

\end{itemize}

\sphinxlineitem{Returns}
\sphinxAtStartPar
data frame with data quality annotation\_df\_quality, stored in the object

\sphinxlineitem{Return type}
\sphinxAtStartPar
pd.DataFrame

\end{description}\end{quote}

\end{fulllineitems}

\index{mlf\_data\_read() (BeeData.BeeData method)@\spxentry{mlf\_data\_read()}\spxextra{BeeData.BeeData method}}

\begin{fulllineitems}
\phantomsection\label{\detokenize{BeeData:BeeData.BeeData.mlf_data_read}}
\pysigstartsignatures
\pysiglinewithargsret{\sphinxbfcode{\sphinxupquote{mlf\_data\_read}}}{}{}
\pysigstopsignatures
\sphinxAtStartPar
Read a file with .mlf extension. The file is expected to have the following format:
\sphinxhyphen{} acoustic file name to be on a new row
\sphinxhyphen{} each column entry to be separated by a tab
\sphinxhyphen{} each tow to end with a new row sign
\sphinxhyphen{} columns \sphinxhyphen{} start, end (in min and seconds format) and label \sphinxhyphen{}bee and nobee
\begin{quote}\begin{description}
\sphinxlineitem{Returns}
\sphinxAtStartPar
a pandas dataframe

\sphinxlineitem{Return type}
\sphinxAtStartPar
pandas.DataFrame

\end{description}\end{quote}

\end{fulllineitems}

\index{split\_acoustic\_data\_sliced() (BeeData.BeeData method)@\spxentry{split\_acoustic\_data\_sliced()}\spxextra{BeeData.BeeData method}}

\begin{fulllineitems}
\phantomsection\label{\detokenize{BeeData:BeeData.BeeData.split_acoustic_data_sliced}}
\pysigstartsignatures
\pysiglinewithargsret{\sphinxbfcode{\sphinxupquote{split\_acoustic\_data\_sliced}}}{}{}
\pysigstopsignatures
\sphinxAtStartPar
Splits the original files based on the annotation data sliced and saves it in the folder
\begin{quote}\begin{description}
\sphinxlineitem{Returns}
\sphinxAtStartPar
wav files

\end{description}\end{quote}

\end{fulllineitems}

\index{time\_slice() (BeeData.BeeData method)@\spxentry{time\_slice()}\spxextra{BeeData.BeeData method}}

\begin{fulllineitems}
\phantomsection\label{\detokenize{BeeData:BeeData.BeeData.time_slice}}
\pysigstartsignatures
\pysiglinewithargsret{\sphinxbfcode{\sphinxupquote{time\_slice}}}{\sphinxparam{\DUrole{n}{nobee}\DUrole{o}{=}\DUrole{default_value}{False}}\sphinxparamcomma \sphinxparam{\DUrole{n}{step}\DUrole{o}{=}\DUrole{default_value}{2000}}\sphinxparamcomma \sphinxparam{\DUrole{n}{start\_sliced\_col\_name}\DUrole{o}{=}\DUrole{default_value}{\textquotesingle{}start\_sliced\textquotesingle{}}}\sphinxparamcomma \sphinxparam{\DUrole{n}{end\_sliced\_col\_name}\DUrole{o}{=}\DUrole{default_value}{\textquotesingle{}end\_sliced\textquotesingle{}}}}{}
\pysigstopsignatures
\sphinxAtStartPar
Function to split the annotation data into smaller segments based on the step size
\begin{quote}\begin{description}
\sphinxlineitem{Parameters}\begin{itemize}
\item {} 
\sphinxAtStartPar
\sphinxstyleliteralstrong{\sphinxupquote{nobee}} (\sphinxstyleliteralemphasis{\sphinxupquote{bool}}) \textendash{} Boolean value to indicate if nobee files to be added in the data set. True means nobee files are included.

\item {} 
\sphinxAtStartPar
\sphinxstyleliteralstrong{\sphinxupquote{step}} (\sphinxstyleliteralemphasis{\sphinxupquote{float}}) \textendash{} time step in miliseconds

\item {} 
\sphinxAtStartPar
\sphinxstyleliteralstrong{\sphinxupquote{start\_sliced\_col\_name}} (\sphinxstyleliteralemphasis{\sphinxupquote{str}}) \textendash{} name of the column for the start sliced

\item {} 
\sphinxAtStartPar
\sphinxstyleliteralstrong{\sphinxupquote{end\_sliced\_col\_name}} (\sphinxstyleliteralemphasis{\sphinxupquote{str}}) \textendash{} name of the column for the end sliced

\end{itemize}

\sphinxlineitem{Returns}
\sphinxAtStartPar
pd.DataFrame with an annotation data which is sliced, stored in self.annotation\_df\_sliced

\sphinxlineitem{Return type}
\sphinxAtStartPar
pd.DataFrame

\end{description}\end{quote}

\end{fulllineitems}


\end{fulllineitems}


\sphinxstepscope


\section{Bee Classification module}
\label{\detokenize{BeeClassification:module-BeeClassification}}\label{\detokenize{BeeClassification:bee-classification-module}}\label{\detokenize{BeeClassification::doc}}\index{module@\spxentry{module}!BeeClassification@\spxentry{BeeClassification}}\index{BeeClassification@\spxentry{BeeClassification}!module@\spxentry{module}}\index{BeeClassification (class in BeeClassification)@\spxentry{BeeClassification}\spxextra{class in BeeClassification}}

\begin{fulllineitems}
\phantomsection\label{\detokenize{BeeClassification:BeeClassification.BeeClassification}}
\pysigstartsignatures
\pysiglinewithargsret{\sphinxbfcode{\sphinxupquote{class\DUrole{w}{ }}}\sphinxcode{\sphinxupquote{BeeClassification.}}\sphinxbfcode{\sphinxupquote{BeeClassification}}}{\sphinxparam{\DUrole{n}{annotation\_path}\DUrole{o}{=}\DUrole{default_value}{\textquotesingle{}beeAnnotations\_enhanced.csv\textquotesingle{}}}\sphinxparamcomma \sphinxparam{\DUrole{n}{annotation\_dtypes\_path}\DUrole{o}{=}\DUrole{default_value}{\textquotesingle{}annotation\_data\_types.csv\textquotesingle{}}}\sphinxparamcomma \sphinxparam{\DUrole{n}{x\_col}\DUrole{o}{=}\DUrole{default_value}{\textquotesingle{}index\textquotesingle{}}}\sphinxparamcomma \sphinxparam{\DUrole{n}{y\_col}\DUrole{o}{=}\DUrole{default_value}{\textquotesingle{}label\textquotesingle{}}}\sphinxparamcomma \sphinxparam{\DUrole{n}{bee\_col}\DUrole{o}{=}\DUrole{default_value}{\textquotesingle{}label\textquotesingle{}}}\sphinxparamcomma \sphinxparam{\DUrole{n}{logname}\DUrole{o}{=}\DUrole{default_value}{\textquotesingle{}bee.log\textquotesingle{}}}\sphinxparamcomma \sphinxparam{\DUrole{n}{acoustic\_folder}\DUrole{o}{=}\DUrole{default_value}{\textquotesingle{}data/SplitData/\textquotesingle{}}}\sphinxparamcomma \sphinxparam{\DUrole{n}{augment\_folder}\DUrole{o}{=}\DUrole{default_value}{\textquotesingle{}data/augment/\textquotesingle{}}}\sphinxparamcomma \sphinxparam{\DUrole{n}{datadict\_folder}\DUrole{o}{=}\DUrole{default_value}{\textquotesingle{}data/DataDict/\textquotesingle{}}}}{}
\pysigstopsignatures
\sphinxAtStartPar
Bases: \sphinxcode{\sphinxupquote{object}}

\sphinxAtStartPar
BeeNotBee class ….
Class to conduct modelling based on labeled audio data, using HuggingFace transformers and Random Forest
\begin{quote}\begin{description}
\sphinxlineitem{Parameters}\begin{itemize}
\item {} 
\sphinxAtStartPar
\sphinxstyleliteralstrong{\sphinxupquote{annotation\_path}} (\sphinxstyleliteralemphasis{\sphinxupquote{str}}) \textendash{} path to the annotation data which holds the labeled audio data

\item {} 
\sphinxAtStartPar
\sphinxstyleliteralstrong{\sphinxupquote{y\_col}} (\sphinxstyleliteralemphasis{\sphinxupquote{str}}) \textendash{} column label for the dependent variable in the annotation file, the label

\item {} 
\sphinxAtStartPar
\sphinxstyleliteralstrong{\sphinxupquote{x\_col}} (\sphinxstyleliteralemphasis{\sphinxupquote{str}}) \textendash{} column label for the independent variable in the annotation file. It is used to read the wav file.

\item {} 
\sphinxAtStartPar
\sphinxstyleliteralstrong{\sphinxupquote{acoustic\_folder}} (\sphinxstyleliteralemphasis{\sphinxupquote{str}}) \textendash{} folder name where acoustic files are stored

\item {} 
\sphinxAtStartPar
\sphinxstyleliteralstrong{\sphinxupquote{acoustic\_folder}} \textendash{} folder name where the split files are stored

\item {} 
\sphinxAtStartPar
\sphinxstyleliteralstrong{\sphinxupquote{augment\_folder}} (\sphinxstyleliteralemphasis{\sphinxupquote{str}}) \textendash{} folder name where the augment files are stored

\item {} 
\sphinxAtStartPar
\sphinxstyleliteralstrong{\sphinxupquote{datadict\_folder}} \textendash{} folder name where the datadict files are stored

\item {} 
\sphinxAtStartPar
\sphinxstyleliteralstrong{\sphinxupquote{logname}} (\sphinxstyleliteralemphasis{\sphinxupquote{str}}) \textendash{} path to the log file

\end{itemize}

\sphinxlineitem{Bee\_col}
\sphinxAtStartPar
column which indicates the dependent variable in the annotation. In the beginning bothe bee\_col and y\_col are the same, but y\_col can be updated during specific actions.

\sphinxlineitem{Returns}
\sphinxAtStartPar
BeeClassification object

\sphinxlineitem{Return type}
\sphinxAtStartPar
{\hyperref[\detokenize{BeeClassification:BeeClassification.BeeClassification}]{\sphinxcrossref{BeeClassification}}}

\end{description}\end{quote}
\index{accuracy\_metrics() (BeeClassification.BeeClassification method)@\spxentry{accuracy\_metrics()}\spxextra{BeeClassification.BeeClassification method}}

\begin{fulllineitems}
\phantomsection\label{\detokenize{BeeClassification:BeeClassification.BeeClassification.accuracy_metrics}}
\pysigstartsignatures
\pysiglinewithargsret{\sphinxbfcode{\sphinxupquote{accuracy\_metrics}}}{\sphinxparam{\DUrole{n}{y\_pred}}}{}
\pysigstopsignatures
\sphinxAtStartPar
Provide accuracy metrics to compare the different models
\begin{quote}\begin{description}
\sphinxlineitem{Parameters}
\sphinxAtStartPar
\sphinxstyleliteralstrong{\sphinxupquote{y\_pred}} (\sphinxstyleliteralemphasis{\sphinxupquote{list}}) \textendash{} predicted dependent values

\sphinxlineitem{Returns}
\sphinxAtStartPar
accuracy, precision, recall

\sphinxlineitem{Return type}
\sphinxAtStartPar
tuple

\end{description}\end{quote}

\end{fulllineitems}

\index{best\_model() (BeeClassification.BeeClassification method)@\spxentry{best\_model()}\spxextra{BeeClassification.BeeClassification method}}

\begin{fulllineitems}
\phantomsection\label{\detokenize{BeeClassification:BeeClassification.BeeClassification.best_model}}
\pysigstartsignatures
\pysiglinewithargsret{\sphinxbfcode{\sphinxupquote{best\_model}}}{\sphinxparam{\DUrole{n}{model}}\sphinxparamcomma \sphinxparam{\DUrole{n}{param\_dist}}}{}
\pysigstopsignatures
\sphinxAtStartPar
Identify the best model after tuning the hyperparameters
\begin{quote}\begin{description}
\sphinxlineitem{Parameters}\begin{itemize}
\item {} 
\sphinxAtStartPar
\sphinxstyleliteralstrong{\sphinxupquote{model}} (\sphinxstyleliteralemphasis{\sphinxupquote{Any}}) \textendash{} an initiated machine learning model

\item {} 
\sphinxAtStartPar
\sphinxstyleliteralstrong{\sphinxupquote{param\_dist}} (\sphinxstyleliteralemphasis{\sphinxupquote{dict}}) \textendash{} a dictionary with the parameters and their respective ranges for the tuning

\end{itemize}

\sphinxlineitem{Returns}
\sphinxAtStartPar
RandomizedSearchCV object

\sphinxlineitem{Return type}
\sphinxAtStartPar
RandomizedSearchCV

\end{description}\end{quote}

\end{fulllineitems}

\index{data\_augmentation\_df() (BeeClassification.BeeClassification method)@\spxentry{data\_augmentation\_df()}\spxextra{BeeClassification.BeeClassification method}}

\begin{fulllineitems}
\phantomsection\label{\detokenize{BeeClassification:BeeClassification.BeeClassification.data_augmentation_df}}
\pysigstartsignatures
\pysiglinewithargsret{\sphinxbfcode{\sphinxupquote{data\_augmentation\_df}}}{\sphinxparam{\DUrole{n}{N}\DUrole{o}{=}\DUrole{default_value}{1}}}{}
\pysigstopsignatures
\sphinxAtStartPar
A function which augments the train data and saves it in the augmented folder (initially cleans the folder); replaces the train data by adding the augmented files information; stores the information in augmented\_df; and saves the names of the augmented files in augmented\_files.
\begin{quote}\begin{description}
\sphinxlineitem{Parameters}
\sphinxAtStartPar
\sphinxstyleliteralstrong{\sphinxupquote{N}} (\sphinxstyleliteralemphasis{\sphinxupquote{int}}) \textendash{} the number of times the augmentation process should happen

\sphinxlineitem{Returns}
\sphinxAtStartPar
the augmented data is saved

\end{description}\end{quote}

\end{fulllineitems}

\index{data\_augmentation\_row() (BeeClassification.BeeClassification method)@\spxentry{data\_augmentation\_row()}\spxextra{BeeClassification.BeeClassification method}}

\begin{fulllineitems}
\phantomsection\label{\detokenize{BeeClassification:BeeClassification.BeeClassification.data_augmentation_row}}
\pysigstartsignatures
\pysiglinewithargsret{\sphinxbfcode{\sphinxupquote{data\_augmentation\_row}}}{\sphinxparam{\DUrole{n}{arg}}}{}
\pysigstopsignatures
\sphinxAtStartPar
A row\sphinxhyphen{}wise function which augments acoustic data and saves it in the acoustic\_folder.
\begin{quote}\begin{description}
\sphinxlineitem{Parameters}
\sphinxAtStartPar
\sphinxstyleliteralstrong{\sphinxupquote{arg}} (\sphinxstyleliteralemphasis{\sphinxupquote{tuple}}) \textendash{} tuple with first argument the index of each row of a data frame, second argument \sphinxhyphen{} the actual row of the data frame

\sphinxlineitem{Returns}
\sphinxAtStartPar
list of lists with the augmented data file index, the augmented data train index, the label of the augmented file and the original train index. If an issue occurs, only the original train and file indices are returned.

\sphinxlineitem{Return type}
\sphinxAtStartPar
list

\end{description}\end{quote}

\end{fulllineitems}

\index{data\_transformation\_df() (BeeClassification.BeeClassification method)@\spxentry{data\_transformation\_df()}\spxextra{BeeClassification.BeeClassification method}}

\begin{fulllineitems}
\phantomsection\label{\detokenize{BeeClassification:BeeClassification.BeeClassification.data_transformation_df}}
\pysigstartsignatures
\pysiglinewithargsret{\sphinxbfcode{\sphinxupquote{data\_transformation\_df}}}{\sphinxparam{\DUrole{n}{X}}\sphinxparamcomma \sphinxparam{\DUrole{n}{func}}}{}
\pysigstopsignatures
\sphinxAtStartPar
Find the correct file from the annotation data frame and then transform the acoustic data using mfcc or mel spec methods. Store the index from the annotation data frame (the key) and the df index to track the associated y values.
\begin{quote}\begin{description}
\sphinxlineitem{Parameters}\begin{itemize}
\item {} 
\sphinxAtStartPar
\sphinxstyleliteralstrong{\sphinxupquote{X}} (\sphinxstyleliteralemphasis{\sphinxupquote{pandas.DataFrame}}) \textendash{} pandas data frame with the indices of the acoustic files which need to be transformed

\item {} 
\sphinxAtStartPar
\sphinxstyleliteralstrong{\sphinxupquote{func}} (\sphinxstyleliteralemphasis{\sphinxupquote{function}}\sphinxstyleliteralemphasis{\sphinxupquote{ or }}\sphinxstyleliteralemphasis{\sphinxupquote{method}}) \textendash{} function for audio files transformation. One can choose from a list of options {[}‘mfcc’,’mel spec’{]}

\end{itemize}

\sphinxlineitem{Returns}
\sphinxAtStartPar
data frame with the transformed data

\sphinxlineitem{Return type}
\sphinxAtStartPar
pandas.dataFrame

\end{description}\end{quote}

\end{fulllineitems}

\index{data\_transformation\_row() (BeeClassification.BeeClassification method)@\spxentry{data\_transformation\_row()}\spxextra{BeeClassification.BeeClassification method}}

\begin{fulllineitems}
\phantomsection\label{\detokenize{BeeClassification:BeeClassification.BeeClassification.data_transformation_row}}
\pysigstartsignatures
\pysiglinewithargsret{\sphinxbfcode{\sphinxupquote{data\_transformation\_row}}}{\sphinxparam{\DUrole{n}{arg}}}{}
\pysigstopsignatures
\sphinxAtStartPar
A row\sphinxhyphen{}wise function which finds the correct file from the annotation data frame and then transforms the acoustic data with mfcc or mel spec. n\_fft is set to 1000 due to the nature of the audio data.
\begin{quote}\begin{description}
\sphinxlineitem{Parameters}
\sphinxAtStartPar
\sphinxstyleliteralstrong{\sphinxupquote{arg}} (\sphinxstyleliteralemphasis{\sphinxupquote{tuple}}) \textendash{} tuple with first argument the index of each row of a data frame, second argument \sphinxhyphen{} the actual row of the data frame and third argument \sphinxhyphen{} data frame with the dependant variable

\sphinxlineitem{Returns}
\sphinxAtStartPar
list of lists with the transformed data. The non\sphinxhyphen{}existent files are returned as None type.

\sphinxlineitem{Return type}
\sphinxAtStartPar
list

\end{description}\end{quote}

\end{fulllineitems}

\index{dataframe\_to\_datadict() (BeeClassification.BeeClassification method)@\spxentry{dataframe\_to\_datadict()}\spxextra{BeeClassification.BeeClassification method}}

\begin{fulllineitems}
\phantomsection\label{\detokenize{BeeClassification:BeeClassification.BeeClassification.dataframe_to_datadict}}
\pysigstartsignatures
\pysiglinewithargsret{\sphinxbfcode{\sphinxupquote{dataframe\_to\_datadict}}}{\sphinxparam{\DUrole{n}{train\_df}}\sphinxparamcomma \sphinxparam{\DUrole{n}{test\_df}}}{}
\pysigstopsignatures
\sphinxAtStartPar
Converts two data frames (test and train) into a data dict which will be used for HuggingFace transformers.
\begin{quote}\begin{description}
\sphinxlineitem{Parameters}\begin{itemize}
\item {} 
\sphinxAtStartPar
\sphinxstyleliteralstrong{\sphinxupquote{train\_df}} (\sphinxstyleliteralemphasis{\sphinxupquote{pd.DataFrame}}) \textendash{} data frame for the training set

\item {} 
\sphinxAtStartPar
\sphinxstyleliteralstrong{\sphinxupquote{test\_df}} (\sphinxstyleliteralemphasis{\sphinxupquote{pd.DataFrame}}) \textendash{} data frame for the testing set

\end{itemize}

\sphinxlineitem{Returns}
\sphinxAtStartPar
datadict\_data, stored in the object

\sphinxlineitem{Return type}
\sphinxAtStartPar
datadict

\end{description}\end{quote}

\end{fulllineitems}

\index{dataframe\_to\_dataset() (BeeClassification.BeeClassification method)@\spxentry{dataframe\_to\_dataset()}\spxextra{BeeClassification.BeeClassification method}}

\begin{fulllineitems}
\phantomsection\label{\detokenize{BeeClassification:BeeClassification.BeeClassification.dataframe_to_dataset}}
\pysigstartsignatures
\pysiglinewithargsret{\sphinxbfcode{\sphinxupquote{dataframe\_to\_dataset}}}{\sphinxparam{\DUrole{n}{df}}\sphinxparamcomma \sphinxparam{\DUrole{n}{split\_type}}\sphinxparamcomma \sphinxparam{\DUrole{n}{num\_chunks}\DUrole{o}{=}\DUrole{default_value}{10}}}{}
\pysigstopsignatures
\sphinxAtStartPar
Splits a dataframe in specific number of chunks. Converts a pandas dataframe to data dict which is used in hugging face transformers. Then saves those chunks into files, reads them and returns the train data.
\begin{quote}\begin{description}
\sphinxlineitem{Parameters}\begin{itemize}
\item {} 
\sphinxAtStartPar
\sphinxstyleliteralstrong{\sphinxupquote{df}} (\sphinxstyleliteralemphasis{\sphinxupquote{pd.DataFrame}}) \textendash{} pandas data frame which has to be converted

\item {} 
\sphinxAtStartPar
\sphinxstyleliteralstrong{\sphinxupquote{split\_type}} (\sphinxstyleliteralemphasis{\sphinxupquote{str}}) \textendash{} ‘train’ or ‘test’ for training and testing sets

\item {} 
\sphinxAtStartPar
\sphinxstyleliteralstrong{\sphinxupquote{num\_chunks}} \textendash{} number of chunks to split the data

\end{itemize}

\sphinxlineitem{Rtype num\_chunks}
\sphinxAtStartPar
int

\sphinxlineitem{Returns}
\sphinxAtStartPar
Dataset

\sphinxlineitem{Return type}
\sphinxAtStartPar
Dataset

\end{description}\end{quote}

\end{fulllineitems}

\index{dataframe\_to\_dataset\_split\_save() (BeeClassification.BeeClassification method)@\spxentry{dataframe\_to\_dataset\_split\_save()}\spxextra{BeeClassification.BeeClassification method}}

\begin{fulllineitems}
\phantomsection\label{\detokenize{BeeClassification:BeeClassification.BeeClassification.dataframe_to_dataset_split_save}}
\pysigstartsignatures
\pysiglinewithargsret{\sphinxbfcode{\sphinxupquote{dataframe\_to\_dataset\_split\_save}}}{\sphinxparam{\DUrole{n}{df}}\sphinxparamcomma \sphinxparam{\DUrole{n}{split\_type}}\sphinxparamcomma \sphinxparam{\DUrole{n}{file\_name}}}{}
\pysigstopsignatures
\sphinxAtStartPar
Converts a pandas dataframe to data dict which is used in hugging face transformers. Then saves the data to the datadict\_folder.
\begin{quote}\begin{description}
\sphinxlineitem{Parameters}\begin{itemize}
\item {} 
\sphinxAtStartPar
\sphinxstyleliteralstrong{\sphinxupquote{df}} (\sphinxstyleliteralemphasis{\sphinxupquote{pd.DataFrame}}) \textendash{} pandas data frame which has to be converted

\item {} 
\sphinxAtStartPar
\sphinxstyleliteralstrong{\sphinxupquote{split\_type}} (\sphinxstyleliteralemphasis{\sphinxupquote{str}}) \textendash{} ‘train’ or ‘test’ for training and testing sets

\item {} 
\sphinxAtStartPar
\sphinxstyleliteralstrong{\sphinxupquote{file\_name}} (\sphinxstyleliteralemphasis{\sphinxupquote{str}}) \textendash{} name of the file to be saved

\end{itemize}

\end{description}\end{quote}

\end{fulllineitems}

\index{file\_read() (BeeClassification.BeeClassification method)@\spxentry{file\_read()}\spxextra{BeeClassification.BeeClassification method}}

\begin{fulllineitems}
\phantomsection\label{\detokenize{BeeClassification:BeeClassification.BeeClassification.file_read}}
\pysigstartsignatures
\pysiglinewithargsret{\sphinxbfcode{\sphinxupquote{file\_read}}}{\sphinxparam{\DUrole{n}{file\_index}}\sphinxparamcomma \sphinxparam{\DUrole{n}{output\_file\_name}\DUrole{o}{=}\DUrole{default_value}{False}}}{}
\pysigstopsignatures
\sphinxAtStartPar
Read a wav file from the acoustic\_folder where the name of the file has an index.
\begin{quote}\begin{description}
\sphinxlineitem{Parameters}\begin{itemize}
\item {} 
\sphinxAtStartPar
\sphinxstyleliteralstrong{\sphinxupquote{file\_index}} (\sphinxstyleliteralemphasis{\sphinxupquote{int}}) \textendash{} index of the file we need to search for

\item {} 
\sphinxAtStartPar
\sphinxstyleliteralstrong{\sphinxupquote{output\_file\_name}} (\sphinxstyleliteralemphasis{\sphinxupquote{bool}}) \textendash{} boolean indicating if file name should be outputed

\end{itemize}

\sphinxlineitem{Returns}
\sphinxAtStartPar
file name, samples and sample rate arrays

\sphinxlineitem{Return type}
\sphinxAtStartPar
string, numpy.ndarray and int

\end{description}\end{quote}

\end{fulllineitems}

\index{misclassified\_analysis() (BeeClassification.BeeClassification method)@\spxentry{misclassified\_analysis()}\spxextra{BeeClassification.BeeClassification method}}

\begin{fulllineitems}
\phantomsection\label{\detokenize{BeeClassification:BeeClassification.BeeClassification.misclassified_analysis}}
\pysigstartsignatures
\pysiglinewithargsret{\sphinxbfcode{\sphinxupquote{misclassified\_analysis}}}{\sphinxparam{\DUrole{n}{y\_pred}}}{}
\pysigstopsignatures
\sphinxAtStartPar
Misclassification analysis to understand where the model miscalculates and if any pattern can be found
\begin{quote}\begin{description}
\sphinxlineitem{Parameters}
\sphinxAtStartPar
\sphinxstyleliteralstrong{\sphinxupquote{y\_pred}} (\sphinxstyleliteralemphasis{\sphinxupquote{list}}) \textendash{} predicted dependent values

\sphinxlineitem{Returns}
\sphinxAtStartPar
misclassified values

\sphinxlineitem{Return type}
\sphinxAtStartPar
pandas.Series

\end{description}\end{quote}

\end{fulllineitems}

\index{model\_results() (BeeClassification.BeeClassification method)@\spxentry{model\_results()}\spxextra{BeeClassification.BeeClassification method}}

\begin{fulllineitems}
\phantomsection\label{\detokenize{BeeClassification:BeeClassification.BeeClassification.model_results}}
\pysigstartsignatures
\pysiglinewithargsret{\sphinxbfcode{\sphinxupquote{model\_results}}}{\sphinxparam{\DUrole{n}{model}}\sphinxparamcomma \sphinxparam{\DUrole{n}{param\_dist}}\sphinxparamcomma \sphinxparam{\DUrole{n}{func}\DUrole{o}{=}\DUrole{default_value}{\textquotesingle{}mfcc\textquotesingle{}}}\sphinxparamcomma \sphinxparam{\DUrole{n}{do\_pca}\DUrole{o}{=}\DUrole{default_value}{True}}}{}
\pysigstopsignatures
\sphinxAtStartPar
Provide a full picture of the model performance and accuracy
\begin{quote}\begin{description}
\sphinxlineitem{Parameters}\begin{itemize}
\item {} 
\sphinxAtStartPar
\sphinxstyleliteralstrong{\sphinxupquote{model}} (\sphinxstyleliteralemphasis{\sphinxupquote{Any}}) \textendash{} an initiated machine learning model such as Random Forest

\item {} 
\sphinxAtStartPar
\sphinxstyleliteralstrong{\sphinxupquote{param\_dist}} (\sphinxstyleliteralemphasis{\sphinxupquote{dict}}) \textendash{} a dictionary with the parameters and their respective ranges for the tuning

\item {} 
\sphinxAtStartPar
\sphinxstyleliteralstrong{\sphinxupquote{func}} (\sphinxstyleliteralemphasis{\sphinxupquote{str}}) \textendash{} function to transform the input variables. Possible values are ‘mfcc’ and ‘mel spec’

\item {} 
\sphinxAtStartPar
\sphinxstyleliteralstrong{\sphinxupquote{do\_pca}} (\sphinxstyleliteralemphasis{\sphinxupquote{bool}}) \textendash{} whether to run pca on the data and take the first two dimensions

\end{itemize}

\sphinxlineitem{Returns}
\sphinxAtStartPar
the best model with its accuracy metrics, misclassified analysis and pca explained variance (do\_pca = True)

\sphinxlineitem{Return type}
\sphinxAtStartPar
tuple

\end{description}\end{quote}

\end{fulllineitems}

\index{new\_y\_label\_creation() (BeeClassification.BeeClassification method)@\spxentry{new\_y\_label\_creation()}\spxextra{BeeClassification.BeeClassification method}}

\begin{fulllineitems}
\phantomsection\label{\detokenize{BeeClassification:BeeClassification.BeeClassification.new_y_label_creation}}
\pysigstartsignatures
\pysiglinewithargsret{\sphinxbfcode{\sphinxupquote{new\_y\_label\_creation}}}{\sphinxparam{\DUrole{n}{old\_col}\DUrole{o}{=}\DUrole{default_value}{{[}\textquotesingle{}missing queen\textquotesingle{}, \textquotesingle{}queen\textquotesingle{}, \textquotesingle{}active day\textquotesingle{}, \textquotesingle{}swarming\textquotesingle{}{]}}}\sphinxparamcomma \sphinxparam{\DUrole{n}{new\_col}\DUrole{o}{=}\DUrole{default_value}{\textquotesingle{}action\textquotesingle{}}}}{}
\pysigstopsignatures
\sphinxAtStartPar
Transform boolean columns into one column.
\begin{quote}\begin{description}
\sphinxlineitem{Parameters}\begin{itemize}
\item {} 
\sphinxAtStartPar
\sphinxstyleliteralstrong{\sphinxupquote{old\_col}} (\sphinxstyleliteralemphasis{\sphinxupquote{list}}) \textendash{} list of column names

\item {} 
\sphinxAtStartPar
\sphinxstyleliteralstrong{\sphinxupquote{new\_col}} (\sphinxstyleliteralemphasis{\sphinxupquote{str}}) \textendash{} new column name

\end{itemize}

\sphinxlineitem{Returns}
\sphinxAtStartPar
updated annotation dataframe with a new column and updated y\_col in the class

\end{description}\end{quote}

\end{fulllineitems}

\index{random\_forest\_results() (BeeClassification.BeeClassification method)@\spxentry{random\_forest\_results()}\spxextra{BeeClassification.BeeClassification method}}

\begin{fulllineitems}
\phantomsection\label{\detokenize{BeeClassification:BeeClassification.BeeClassification.random_forest_results}}
\pysigstartsignatures
\pysiglinewithargsret{\sphinxbfcode{\sphinxupquote{random\_forest\_results}}}{\sphinxparam{\DUrole{n}{func}\DUrole{o}{=}\DUrole{default_value}{\textquotesingle{}mfcc\textquotesingle{}}}\sphinxparamcomma \sphinxparam{\DUrole{n}{do\_pca}\DUrole{o}{=}\DUrole{default_value}{True}}}{}
\pysigstopsignatures
\sphinxAtStartPar
Run Random Forest and conduct hyperparameter tuning, accuracy measurement and feature importance

\sphinxAtStartPar
:param func:function to transform the input variables. Possible values are ‘mfcc’ and ‘mel spec’
:type func: str
:param do\_pca: whether to run pca on the data and take the first two dimensions
:type do\_pca: bool
:return: accuracy, precision, recall, misclassified analysis, pca variance (do\_pca=True) and feature importance
:rtype: tuple

\end{fulllineitems}

\index{read\_annotation\_csv() (BeeClassification.BeeClassification method)@\spxentry{read\_annotation\_csv()}\spxextra{BeeClassification.BeeClassification method}}

\begin{fulllineitems}
\phantomsection\label{\detokenize{BeeClassification:BeeClassification.BeeClassification.read_annotation_csv}}
\pysigstartsignatures
\pysiglinewithargsret{\sphinxbfcode{\sphinxupquote{read\_annotation\_csv}}}{}{}
\pysigstopsignatures
\sphinxAtStartPar
Read annotation data
\begin{quote}\begin{description}
\sphinxlineitem{Returns}
\sphinxAtStartPar
pandas data frame with the annotations

\end{description}\end{quote}

\end{fulllineitems}

\index{split\_annotation\_data() (BeeClassification.BeeClassification method)@\spxentry{split\_annotation\_data()}\spxextra{BeeClassification.BeeClassification method}}

\begin{fulllineitems}
\phantomsection\label{\detokenize{BeeClassification:BeeClassification.BeeClassification.split_annotation_data}}
\pysigstartsignatures
\pysiglinewithargsret{\sphinxbfcode{\sphinxupquote{split\_annotation\_data}}}{\sphinxparam{\DUrole{n}{perc}\DUrole{o}{=}\DUrole{default_value}{0.25}}\sphinxparamcomma \sphinxparam{\DUrole{n}{stratified}\DUrole{o}{=}\DUrole{default_value}{True}}}{}
\pysigstopsignatures
\sphinxAtStartPar
Split the annotation data into train and test based on the y\_col and x\_col values. Save the csv files.
:param perc: test split percentage, a number between 0.0 and 1.0
:type perc: float
:param stratified: Boolean value to indicate if the split should be stratified
:type stratified: bool
:return:annotation\_df\_updated with the final data which is split. X train, X test, y train and y test pandas data frames

\end{fulllineitems}

\index{transformer\_classification() (BeeClassification.BeeClassification method)@\spxentry{transformer\_classification()}\spxextra{BeeClassification.BeeClassification method}}

\begin{fulllineitems}
\phantomsection\label{\detokenize{BeeClassification:BeeClassification.BeeClassification.transformer_classification}}
\pysigstartsignatures
\pysiglinewithargsret{\sphinxbfcode{\sphinxupquote{transformer\_classification}}}{\sphinxparam{\DUrole{n}{data}}\sphinxparamcomma \sphinxparam{\DUrole{n}{max\_duration}\DUrole{o}{=}\DUrole{default_value}{5}}\sphinxparamcomma \sphinxparam{\DUrole{n}{model\_id}\DUrole{o}{=}\DUrole{default_value}{\textquotesingle{}facebook/hubert\sphinxhyphen{}base\sphinxhyphen{}ls960\textquotesingle{}}}\sphinxparamcomma \sphinxparam{\DUrole{n}{batch\_size}\DUrole{o}{=}\DUrole{default_value}{8}}\sphinxparamcomma \sphinxparam{\DUrole{n}{gradient\_accumulation\_steps}\DUrole{o}{=}\DUrole{default_value}{4}}\sphinxparamcomma \sphinxparam{\DUrole{n}{num\_train\_epochs}\DUrole{o}{=}\DUrole{default_value}{10}}\sphinxparamcomma \sphinxparam{\DUrole{n}{warmup\_ratio}\DUrole{o}{=}\DUrole{default_value}{0.1}}\sphinxparamcomma \sphinxparam{\DUrole{n}{logging\_steps}\DUrole{o}{=}\DUrole{default_value}{10}}\sphinxparamcomma \sphinxparam{\DUrole{n}{learning\_rate}\DUrole{o}{=}\DUrole{default_value}{3e\sphinxhyphen{}05}}\sphinxparamcomma \sphinxparam{\DUrole{n}{name}\DUrole{o}{=}\DUrole{default_value}{\textquotesingle{}finetuned\sphinxhyphen{}bee\textquotesingle{}}}\sphinxparamcomma \sphinxparam{\DUrole{n}{wandb}\DUrole{o}{=}\DUrole{default_value}{True}}}{}
\pysigstopsignatures
\sphinxAtStartPar
Execute huggingface transformer pre\sphinxhyphen{}trained classification model for audio data. It has been integrated with Weights \& Bises for further development and monitoring.
\begin{quote}\begin{description}
\sphinxlineitem{Parameters}\begin{itemize}
\item {} 
\sphinxAtStartPar
\sphinxstyleliteralstrong{\sphinxupquote{data}} (\sphinxstyleliteralemphasis{\sphinxupquote{DataDict}}) \textendash{} DataDict for audio data with train and test

\item {} 
\sphinxAtStartPar
\sphinxstyleliteralstrong{\sphinxupquote{max\_duration}} (\sphinxstyleliteralemphasis{\sphinxupquote{int}}) \textendash{} maximum duration of the data file

\item {} 
\sphinxAtStartPar
\sphinxstyleliteralstrong{\sphinxupquote{model\_id}} (\sphinxstyleliteralemphasis{\sphinxupquote{str}}) \textendash{} the name of the HiggingFace model

\item {} 
\sphinxAtStartPar
\sphinxstyleliteralstrong{\sphinxupquote{batch\_size}} (\sphinxstyleliteralemphasis{\sphinxupquote{int}}) \textendash{} The batch size per GPU/XPU/TPU/MPS/NPU core/CPU for training/testing.

\item {} 
\sphinxAtStartPar
\sphinxstyleliteralstrong{\sphinxupquote{gradient\_accumulation\_steps}} (\sphinxstyleliteralemphasis{\sphinxupquote{int}}) \textendash{} Number of updates steps to accumulate the gradients for, before performing a backward/update pass.

\item {} 
\sphinxAtStartPar
\sphinxstyleliteralstrong{\sphinxupquote{num\_train\_epochs}} (\sphinxstyleliteralemphasis{\sphinxupquote{float}}) \textendash{} Total number of training epochs to perform (if not an integer, will perform the decimal part percents of the last epoch before stopping training).

\item {} 
\sphinxAtStartPar
\sphinxstyleliteralstrong{\sphinxupquote{warmup\_ratio}} (\sphinxstyleliteralemphasis{\sphinxupquote{float}}) \textendash{} Ratio of total training steps used for a linear warmup from 0 to learning\_rate.

\item {} 
\sphinxAtStartPar
\sphinxstyleliteralstrong{\sphinxupquote{logging\_steps}} (\sphinxstyleliteralemphasis{\sphinxupquote{float}}) \textendash{} Number of update steps between two logs if logging\_strategy=”steps”. Should be an integer or a float in range {[}0,1). If smaller than 1, will be interpreted as ratio of total training steps.

\item {} 
\sphinxAtStartPar
\sphinxstyleliteralstrong{\sphinxupquote{learning\_rate}} (\sphinxstyleliteralemphasis{\sphinxupquote{float}}) \textendash{} The initial learning rate.

\item {} 
\sphinxAtStartPar
\sphinxstyleliteralstrong{\sphinxupquote{name}} (\sphinxstyleliteralemphasis{\sphinxupquote{str}}) \textendash{} name of the newly created model

\item {} 
\sphinxAtStartPar
\sphinxstyleliteralstrong{\sphinxupquote{wandb}} (\sphinxstyleliteralemphasis{\sphinxupquote{bool}}) \textendash{} indicates if the is Weights \& Biases connected profile.

\end{itemize}

\sphinxlineitem{Returns}
\sphinxAtStartPar
trained model

\sphinxlineitem{Return type}
\sphinxAtStartPar
HuggingFace.models

\end{description}\end{quote}

\end{fulllineitems}

\index{validate\_annotation\_csv() (BeeClassification.BeeClassification method)@\spxentry{validate\_annotation\_csv()}\spxextra{BeeClassification.BeeClassification method}}

\begin{fulllineitems}
\phantomsection\label{\detokenize{BeeClassification:BeeClassification.BeeClassification.validate_annotation_csv}}
\pysigstartsignatures
\pysiglinewithargsret{\sphinxbfcode{\sphinxupquote{validate\_annotation\_csv}}}{}{}
\pysigstopsignatures
\sphinxAtStartPar
”
Validate annotation data
\begin{quote}\begin{description}
\sphinxlineitem{Returns}
\sphinxAtStartPar
warning if the data is as expected

\end{description}\end{quote}

\end{fulllineitems}


\end{fulllineitems}


\sphinxstepscope


\section{Auxiliary Functions module}
\label{\detokenize{auxilary_functions:module-auxilary_functions}}\label{\detokenize{auxilary_functions:auxiliary-functions-module}}\label{\detokenize{auxilary_functions::doc}}\index{module@\spxentry{module}!auxilary\_functions@\spxentry{auxilary\_functions}}\index{auxilary\_functions@\spxentry{auxilary\_functions}!module@\spxentry{module}}\index{citations() (in module auxilary\_functions)@\spxentry{citations()}\spxextra{in module auxilary\_functions}}

\begin{fulllineitems}
\phantomsection\label{\detokenize{auxilary_functions:auxilary_functions.citations}}
\pysigstartsignatures
\pysiglinewithargsret{\sphinxcode{\sphinxupquote{auxilary\_functions.}}\sphinxbfcode{\sphinxupquote{citations}}}{\sphinxparam{\DUrole{n}{x}}}{}
\pysigstopsignatures
\sphinxAtStartPar
Function to extract the number of citations from a string
\begin{quote}\begin{description}
\sphinxlineitem{Parameters}
\sphinxAtStartPar
\sphinxstyleliteralstrong{\sphinxupquote{x}} (\sphinxstyleliteralemphasis{\sphinxupquote{str}}) \textendash{} string which has a key word \sphinxhyphen{} Citations to extarct the info from

\sphinxlineitem{Returns}
\sphinxAtStartPar
Number of citations

\sphinxlineitem{Return type}
\sphinxAtStartPar
int

\end{description}\end{quote}

\end{fulllineitems}

\index{clean\_directory() (in module auxilary\_functions)@\spxentry{clean\_directory()}\spxextra{in module auxilary\_functions}}

\begin{fulllineitems}
\phantomsection\label{\detokenize{auxilary_functions:auxilary_functions.clean_directory}}
\pysigstartsignatures
\pysiglinewithargsret{\sphinxcode{\sphinxupquote{auxilary\_functions.}}\sphinxbfcode{\sphinxupquote{clean\_directory}}}{\sphinxparam{\DUrole{n}{path}}\sphinxparamcomma \sphinxparam{\DUrole{n}{folder}\DUrole{o}{=}\DUrole{default_value}{False}}}{}
\pysigstopsignatures
\sphinxAtStartPar
Clean the directory
\begin{quote}\begin{description}
\sphinxlineitem{Parameters}\begin{itemize}
\item {} 
\sphinxAtStartPar
\sphinxstyleliteralstrong{\sphinxupquote{path}} \textendash{} the directory which needs to be cleaned

\item {} 
\sphinxAtStartPar
\sphinxstyleliteralstrong{\sphinxupquote{folder}} (\sphinxstyleliteralemphasis{\sphinxupquote{bool}}) \textendash{} specifies if we need to remove folders

\end{itemize}

\end{description}\end{quote}

\end{fulllineitems}

\index{compute\_metrics() (in module auxilary\_functions)@\spxentry{compute\_metrics()}\spxextra{in module auxilary\_functions}}

\begin{fulllineitems}
\phantomsection\label{\detokenize{auxilary_functions:auxilary_functions.compute_metrics}}
\pysigstartsignatures
\pysiglinewithargsret{\sphinxcode{\sphinxupquote{auxilary\_functions.}}\sphinxbfcode{\sphinxupquote{compute\_metrics}}}{\sphinxparam{\DUrole{n}{eval\_pred}}}{}
\pysigstopsignatures
\sphinxAtStartPar
Computes accuracy on a batch of predictions
\begin{quote}\begin{description}
\sphinxlineitem{Parameters}
\sphinxAtStartPar
\sphinxstyleliteralstrong{\sphinxupquote{eval\_pred}} (\sphinxstyleliteralemphasis{\sphinxupquote{array like}}) \textendash{} predictions returned by the HuggingFace model

\sphinxlineitem{Returns}
\sphinxAtStartPar
the computed evaluation metric

\sphinxlineitem{Return type}
\sphinxAtStartPar
float

\end{description}\end{quote}

\end{fulllineitems}

\index{cos\_func() (in module auxilary\_functions)@\spxentry{cos\_func()}\spxextra{in module auxilary\_functions}}

\begin{fulllineitems}
\phantomsection\label{\detokenize{auxilary_functions:auxilary_functions.cos_func}}
\pysigstartsignatures
\pysiglinewithargsret{\sphinxcode{\sphinxupquote{auxilary\_functions.}}\sphinxbfcode{\sphinxupquote{cos\_func}}}{\sphinxparam{\DUrole{n}{v1}}\sphinxparamcomma \sphinxparam{\DUrole{n}{v2}}}{}
\pysigstopsignatures
\sphinxAtStartPar
Function to calculate the cosine similiarity between two array\sphinxhyphen{}like variables
\begin{quote}\begin{description}
\sphinxlineitem{Parameters}\begin{itemize}
\item {} 
\sphinxAtStartPar
\sphinxstyleliteralstrong{\sphinxupquote{v1}} (\sphinxstyleliteralemphasis{\sphinxupquote{array\sphinxhyphen{}like}}) \textendash{} array\sphinxhyphen{}like variable

\item {} 
\sphinxAtStartPar
\sphinxstyleliteralstrong{\sphinxupquote{v2}} (\sphinxstyleliteralemphasis{\sphinxupquote{array\sphinxhyphen{}like}}) \textendash{} array\sphinxhyphen{}like variable

\end{itemize}

\sphinxlineitem{Returns}
\sphinxAtStartPar
cosine similarity between teh two

\sphinxlineitem{Return type}
\sphinxAtStartPar
float

\end{description}\end{quote}

\end{fulllineitems}

\index{cos\_sim\_func() (in module auxilary\_functions)@\spxentry{cos\_sim\_func()}\spxextra{in module auxilary\_functions}}

\begin{fulllineitems}
\phantomsection\label{\detokenize{auxilary_functions:auxilary_functions.cos_sim_func}}
\pysigstartsignatures
\pysiglinewithargsret{\sphinxcode{\sphinxupquote{auxilary\_functions.}}\sphinxbfcode{\sphinxupquote{cos\_sim\_func}}}{\sphinxparam{\DUrole{n}{pair}}\sphinxparamcomma \sphinxparam{\DUrole{n}{embedding\_list}}}{}
\pysigstopsignatures
\sphinxAtStartPar
Function to calculate the similarity between two embeddings based on cosine similarity
\begin{quote}\begin{description}
\sphinxlineitem{Parameters}\begin{itemize}
\item {} 
\sphinxAtStartPar
\sphinxstyleliteralstrong{\sphinxupquote{pair}} (\sphinxstyleliteralemphasis{\sphinxupquote{tuple}}) \textendash{} tuple with the index of the two embeddings

\item {} 
\sphinxAtStartPar
\sphinxstyleliteralstrong{\sphinxupquote{embedding\_list}} (\sphinxstyleliteralemphasis{\sphinxupquote{array\sphinxhyphen{}like}}) \textendash{} list of the embedding vectors

\end{itemize}

\sphinxlineitem{Returns}
\sphinxAtStartPar
dictionary with pair0, pair1 and cos

\sphinxlineitem{Return type}
\sphinxAtStartPar
dict

\end{description}\end{quote}

\end{fulllineitems}

\index{file\_name\_extract() (in module auxilary\_functions)@\spxentry{file\_name\_extract()}\spxextra{in module auxilary\_functions}}

\begin{fulllineitems}
\phantomsection\label{\detokenize{auxilary_functions:auxilary_functions.file_name_extract}}
\pysigstartsignatures
\pysiglinewithargsret{\sphinxcode{\sphinxupquote{auxilary\_functions.}}\sphinxbfcode{\sphinxupquote{file\_name\_extract}}}{\sphinxparam{\DUrole{n}{row}}\sphinxparamcomma \sphinxparam{\DUrole{n}{column\_name1}}\sphinxparamcomma \sphinxparam{\DUrole{n}{column\_name2}}}{}
\pysigstopsignatures
\sphinxAtStartPar
The goal of this function is to extract the file name and add it to the DF
\begin{quote}\begin{description}
\sphinxlineitem{Parameters}\begin{itemize}
\item {} 
\sphinxAtStartPar
\sphinxstyleliteralstrong{\sphinxupquote{row}} (\sphinxstyleliteralemphasis{\sphinxupquote{pd.Series}}) \textendash{} row from a data frame

\item {} 
\sphinxAtStartPar
\sphinxstyleliteralstrong{\sphinxupquote{column\_name1}} (\sphinxstyleliteralemphasis{\sphinxupquote{str}}) \textendash{} column name which has a label

\item {} 
\sphinxAtStartPar
\sphinxstyleliteralstrong{\sphinxupquote{column\_name2}} (\sphinxstyleliteralemphasis{\sphinxupquote{str}}) \textendash{} another column which has a label

\end{itemize}

\sphinxlineitem{Returns}
\sphinxAtStartPar
extracted file name

\sphinxlineitem{Return type}
\sphinxAtStartPar
str

\end{description}\end{quote}

\end{fulllineitems}

\index{get\_file\_names() (in module auxilary\_functions)@\spxentry{get\_file\_names()}\spxextra{in module auxilary\_functions}}

\begin{fulllineitems}
\phantomsection\label{\detokenize{auxilary_functions:auxilary_functions.get_file_names}}
\pysigstartsignatures
\pysiglinewithargsret{\sphinxcode{\sphinxupquote{auxilary\_functions.}}\sphinxbfcode{\sphinxupquote{get\_file\_names}}}{\sphinxparam{\DUrole{n}{dir\_name}}\sphinxparamcomma \sphinxparam{\DUrole{n}{specific\_extension}\DUrole{o}{=}\DUrole{default_value}{False}}\sphinxparamcomma \sphinxparam{\DUrole{n}{extension}\DUrole{o}{=}\DUrole{default_value}{(\textquotesingle{}.wav\textquotesingle{}, \textquotesingle{}.mp3\textquotesingle{})}}}{}
\pysigstopsignatures
\sphinxAtStartPar
Create a list of files in a specific directory with specified extension
\begin{quote}\begin{description}
\sphinxlineitem{Parameters}\begin{itemize}
\item {} 
\sphinxAtStartPar
\sphinxstyleliteralstrong{\sphinxupquote{dir\_name}} (\sphinxstyleliteralemphasis{\sphinxupquote{str}}) \textendash{} directory which contains the files

\item {} 
\sphinxAtStartPar
\sphinxstyleliteralstrong{\sphinxupquote{specific\_extension}} \textendash{} specify if we would be looking for a specific extension files

\item {} 
\sphinxAtStartPar
\sphinxstyleliteralstrong{\sphinxupquote{extension}} (\sphinxstyleliteralemphasis{\sphinxupquote{tuple}}) \textendash{} tuple with strings indicating the file extension

\end{itemize}

\sphinxlineitem{Returns}
\sphinxAtStartPar
a list of files

\sphinxlineitem{Return type}
\sphinxAtStartPar
list

\end{description}\end{quote}

\end{fulllineitems}

\index{h0() (in module auxilary\_functions)@\spxentry{h0()}\spxextra{in module auxilary\_functions}}

\begin{fulllineitems}
\phantomsection\label{\detokenize{auxilary_functions:auxilary_functions.h0}}
\pysigstartsignatures
\pysiglinewithargsret{\sphinxcode{\sphinxupquote{auxilary\_functions.}}\sphinxbfcode{\sphinxupquote{h0}}}{\sphinxparam{\DUrole{n}{text}}\sphinxparamcomma \sphinxparam{\DUrole{n}{pdf}}\sphinxparamcomma \sphinxparam{\DUrole{n}{x}\DUrole{o}{=}\DUrole{default_value}{20}}}{}
\pysigstopsignatures
\sphinxAtStartPar
Generates h0 title text
\begin{quote}\begin{description}
\sphinxlineitem{Parameters}\begin{itemize}
\item {} 
\sphinxAtStartPar
\sphinxstyleliteralstrong{\sphinxupquote{text}} (\sphinxstyleliteralemphasis{\sphinxupquote{str}}) \textendash{} text for the title

\item {} 
\sphinxAtStartPar
\sphinxstyleliteralstrong{\sphinxupquote{pdf}} (\sphinxstyleliteralemphasis{\sphinxupquote{FPDF}}) \textendash{} FPDF instance

\item {} 
\sphinxAtStartPar
\sphinxstyleliteralstrong{\sphinxupquote{x}} (\sphinxstyleliteralemphasis{\sphinxupquote{int}}) \textendash{} line space after the text

\end{itemize}

\sphinxlineitem{Returns}
\sphinxAtStartPar
text

\end{description}\end{quote}

\end{fulllineitems}

\index{h1() (in module auxilary\_functions)@\spxentry{h1()}\spxextra{in module auxilary\_functions}}

\begin{fulllineitems}
\phantomsection\label{\detokenize{auxilary_functions:auxilary_functions.h1}}
\pysigstartsignatures
\pysiglinewithargsret{\sphinxcode{\sphinxupquote{auxilary\_functions.}}\sphinxbfcode{\sphinxupquote{h1}}}{\sphinxparam{\DUrole{n}{text}}\sphinxparamcomma \sphinxparam{\DUrole{n}{pdf}}\sphinxparamcomma \sphinxparam{\DUrole{n}{x}\DUrole{o}{=}\DUrole{default_value}{10}}}{}
\pysigstopsignatures
\sphinxAtStartPar
Generates h1 title text
\begin{quote}\begin{description}
\sphinxlineitem{Parameters}\begin{itemize}
\item {} 
\sphinxAtStartPar
\sphinxstyleliteralstrong{\sphinxupquote{text}} (\sphinxstyleliteralemphasis{\sphinxupquote{str}}) \textendash{} text for the title

\item {} 
\sphinxAtStartPar
\sphinxstyleliteralstrong{\sphinxupquote{pdf}} (\sphinxstyleliteralemphasis{\sphinxupquote{FPDF}}) \textendash{} FPDF instance

\item {} 
\sphinxAtStartPar
\sphinxstyleliteralstrong{\sphinxupquote{x}} (\sphinxstyleliteralemphasis{\sphinxupquote{int}}) \textendash{} line space after the text

\end{itemize}

\sphinxlineitem{Returns}
\sphinxAtStartPar
text

\end{description}\end{quote}

\end{fulllineitems}

\index{h2() (in module auxilary\_functions)@\spxentry{h2()}\spxextra{in module auxilary\_functions}}

\begin{fulllineitems}
\phantomsection\label{\detokenize{auxilary_functions:auxilary_functions.h2}}
\pysigstartsignatures
\pysiglinewithargsret{\sphinxcode{\sphinxupquote{auxilary\_functions.}}\sphinxbfcode{\sphinxupquote{h2}}}{\sphinxparam{\DUrole{n}{text}}\sphinxparamcomma \sphinxparam{\DUrole{n}{pdf}}\sphinxparamcomma \sphinxparam{\DUrole{n}{x}\DUrole{o}{=}\DUrole{default_value}{10}}}{}
\pysigstopsignatures
\sphinxAtStartPar
Generates h2 title text
\begin{quote}\begin{description}
\sphinxlineitem{Parameters}\begin{itemize}
\item {} 
\sphinxAtStartPar
\sphinxstyleliteralstrong{\sphinxupquote{text}} (\sphinxstyleliteralemphasis{\sphinxupquote{str}}) \textendash{} text for the title

\item {} 
\sphinxAtStartPar
\sphinxstyleliteralstrong{\sphinxupquote{pdf}} (\sphinxstyleliteralemphasis{\sphinxupquote{FPDF}}) \textendash{} FPDF instance

\item {} 
\sphinxAtStartPar
\sphinxstyleliteralstrong{\sphinxupquote{x}} (\sphinxstyleliteralemphasis{\sphinxupquote{int}}) \textendash{} line space after the text

\end{itemize}

\sphinxlineitem{Returns}
\sphinxAtStartPar
text

\end{description}\end{quote}

\end{fulllineitems}

\index{include\_tuple() (in module auxilary\_functions)@\spxentry{include\_tuple()}\spxextra{in module auxilary\_functions}}

\begin{fulllineitems}
\phantomsection\label{\detokenize{auxilary_functions:auxilary_functions.include_tuple}}
\pysigstartsignatures
\pysiglinewithargsret{\sphinxcode{\sphinxupquote{auxilary\_functions.}}\sphinxbfcode{\sphinxupquote{include\_tuple}}}{\sphinxparam{\DUrole{n}{element\_list}}\sphinxparamcomma \sphinxparam{\DUrole{n}{row}}\sphinxparamcomma \sphinxparam{\DUrole{n}{t\_limit}\DUrole{o}{=}\DUrole{default_value}{0.3}}}{}
\pysigstopsignatures
\sphinxAtStartPar
Return the tuple if at least one of the elements is present
\begin{quote}\begin{description}
\sphinxlineitem{Parameters}\begin{itemize}
\item {} 
\sphinxAtStartPar
\sphinxstyleliteralstrong{\sphinxupquote{element\_list}} (\sphinxstyleliteralemphasis{\sphinxupquote{list}}) \textendash{} list of elements to include

\item {} 
\sphinxAtStartPar
\sphinxstyleliteralstrong{\sphinxupquote{row}} (\sphinxstyleliteralemphasis{\sphinxupquote{pd.Series}}) \textendash{} row of a data frame

\item {} 
\sphinxAtStartPar
\sphinxstyleliteralstrong{\sphinxupquote{t\_limit}} (\sphinxstyleliteralemphasis{\sphinxupquote{float}}) \textendash{} threshold to exclude elements if they don’t have good cosine similarity with the include list

\end{itemize}

\sphinxlineitem{Returns}
\sphinxAtStartPar
boolean indicating if the tuple should be included or not

\sphinxlineitem{Return type}
\sphinxAtStartPar
bool

\end{description}\end{quote}

\end{fulllineitems}

\index{normal\_text() (in module auxilary\_functions)@\spxentry{normal\_text()}\spxextra{in module auxilary\_functions}}

\begin{fulllineitems}
\phantomsection\label{\detokenize{auxilary_functions:auxilary_functions.normal_text}}
\pysigstartsignatures
\pysiglinewithargsret{\sphinxcode{\sphinxupquote{auxilary\_functions.}}\sphinxbfcode{\sphinxupquote{normal\_text}}}{\sphinxparam{\DUrole{n}{text}}\sphinxparamcomma \sphinxparam{\DUrole{n}{pdf}}\sphinxparamcomma \sphinxparam{\DUrole{n}{x}\DUrole{o}{=}\DUrole{default_value}{5}}\sphinxparamcomma \sphinxparam{\DUrole{n}{italics}\DUrole{o}{=}\DUrole{default_value}{False}}\sphinxparamcomma \sphinxparam{\DUrole{n}{link\_text}\DUrole{o}{=}\DUrole{default_value}{None}}}{}
\pysigstopsignatures
\sphinxAtStartPar
Generates pdf normal multi\sphinxhyphen{}line text
\begin{quote}\begin{description}
\sphinxlineitem{Parameters}\begin{itemize}
\item {} 
\sphinxAtStartPar
\sphinxstyleliteralstrong{\sphinxupquote{text}} (\sphinxstyleliteralemphasis{\sphinxupquote{str}}) \textendash{} the text

\item {} 
\sphinxAtStartPar
\sphinxstyleliteralstrong{\sphinxupquote{pdf}} (\sphinxstyleliteralemphasis{\sphinxupquote{FPDF}}) \textendash{} FPDF instance

\item {} 
\sphinxAtStartPar
\sphinxstyleliteralstrong{\sphinxupquote{x}} (\sphinxstyleliteralemphasis{\sphinxupquote{int}}) \textendash{} space after the text

\item {} 
\sphinxAtStartPar
\sphinxstyleliteralstrong{\sphinxupquote{italics}} (\sphinxstyleliteralemphasis{\sphinxupquote{bool}}) \textendash{} boolean to indicate if text should be italics

\end{itemize}

\sphinxlineitem{Returns}
\sphinxAtStartPar
pdf generated text

\end{description}\end{quote}

\end{fulllineitems}

\index{pd\_to\_tuple() (in module auxilary\_functions)@\spxentry{pd\_to\_tuple()}\spxextra{in module auxilary\_functions}}

\begin{fulllineitems}
\phantomsection\label{\detokenize{auxilary_functions:auxilary_functions.pd_to_tuple}}
\pysigstartsignatures
\pysiglinewithargsret{\sphinxcode{\sphinxupquote{auxilary\_functions.}}\sphinxbfcode{\sphinxupquote{pd\_to\_tuple}}}{\sphinxparam{\DUrole{n}{df}}\sphinxparamcomma \sphinxparam{\DUrole{n}{col}}}{}
\pysigstopsignatures
\sphinxAtStartPar
Converts pd.dataframe.value\_counts() to tuple for pdf table ingestion.
\begin{quote}\begin{description}
\sphinxlineitem{Parameters}\begin{itemize}
\item {} 
\sphinxAtStartPar
\sphinxstyleliteralstrong{\sphinxupquote{df}} (\sphinxstyleliteralemphasis{\sphinxupquote{pd.DataFrame}}) \textendash{} pandas dataframe

\item {} 
\sphinxAtStartPar
\sphinxstyleliteralstrong{\sphinxupquote{col}} (\sphinxstyleliteralemphasis{\sphinxupquote{str}}) \textendash{} column name for the value\_counts

\end{itemize}

\sphinxlineitem{Returns}
\sphinxAtStartPar
tuple of tuples

\sphinxlineitem{Return type}
\sphinxAtStartPar
tuple

\end{description}\end{quote}

\end{fulllineitems}

\index{pdf\_table() (in module auxilary\_functions)@\spxentry{pdf\_table()}\spxextra{in module auxilary\_functions}}

\begin{fulllineitems}
\phantomsection\label{\detokenize{auxilary_functions:auxilary_functions.pdf_table}}
\pysigstartsignatures
\pysiglinewithargsret{\sphinxcode{\sphinxupquote{auxilary\_functions.}}\sphinxbfcode{\sphinxupquote{pdf\_table}}}{\sphinxparam{\DUrole{n}{table\_data}}\sphinxparamcomma \sphinxparam{\DUrole{n}{pdf}}\sphinxparamcomma \sphinxparam{\DUrole{n}{x}\DUrole{o}{=}\DUrole{default_value}{10}}\sphinxparamcomma \sphinxparam{\DUrole{n}{width}\DUrole{o}{=}\DUrole{default_value}{40}}\sphinxparamcomma \sphinxparam{\DUrole{n}{cols}\DUrole{o}{=}\DUrole{default_value}{(20, 20)}}}{}
\pysigstopsignatures
\sphinxAtStartPar
Genrates table to be printed in the pdf file
\begin{quote}\begin{description}
\sphinxlineitem{Parameters}\begin{itemize}
\item {} 
\sphinxAtStartPar
\sphinxstyleliteralstrong{\sphinxupquote{table\_data}} (\sphinxstyleliteralemphasis{\sphinxupquote{tuple}}) \textendash{} table data in tuple version

\item {} 
\sphinxAtStartPar
\sphinxstyleliteralstrong{\sphinxupquote{pdf}} (\sphinxstyleliteralemphasis{\sphinxupquote{FPDF}}) \textendash{} FPDF instance

\item {} 
\sphinxAtStartPar
\sphinxstyleliteralstrong{\sphinxupquote{x}} (\sphinxstyleliteralemphasis{\sphinxupquote{int}}) \textendash{} lines after the text

\item {} 
\sphinxAtStartPar
\sphinxstyleliteralstrong{\sphinxupquote{width}} (\sphinxstyleliteralemphasis{\sphinxupquote{int}}) \textendash{} width of the table

\item {} 
\sphinxAtStartPar
\sphinxstyleliteralstrong{\sphinxupquote{cols}} (\sphinxstyleliteralemphasis{\sphinxupquote{tuple}}) \textendash{} tuple with teh width of the columns

\end{itemize}

\sphinxlineitem{Returns}
\sphinxAtStartPar
pdf table

\end{description}\end{quote}

\end{fulllineitems}

\index{preprocess\_function() (in module auxilary\_functions)@\spxentry{preprocess\_function()}\spxextra{in module auxilary\_functions}}

\begin{fulllineitems}
\phantomsection\label{\detokenize{auxilary_functions:auxilary_functions.preprocess_function}}
\pysigstartsignatures
\pysiglinewithargsret{\sphinxcode{\sphinxupquote{auxilary\_functions.}}\sphinxbfcode{\sphinxupquote{preprocess\_function}}}{\sphinxparam{\DUrole{n}{examples}}\sphinxparamcomma \sphinxparam{\DUrole{n}{feature\_extractor}}\sphinxparamcomma \sphinxparam{\DUrole{n}{max\_duration}\DUrole{o}{=}\DUrole{default_value}{10}}}{}
\pysigstopsignatures
\sphinxAtStartPar
Function to preprocess the audio data to a predefined sampling rate and duration
\begin{quote}\begin{description}
\sphinxlineitem{Parameters}\begin{itemize}
\item {} 
\sphinxAtStartPar
\sphinxstyleliteralstrong{\sphinxupquote{examples}} (\sphinxstyleliteralemphasis{\sphinxupquote{datadict}}) \textendash{} datadict examples with ‘audio’ key

\item {} 
\sphinxAtStartPar
\sphinxstyleliteralstrong{\sphinxupquote{feature\_extractor}} (\sphinxstyleliteralemphasis{\sphinxupquote{AutoFeatureExtractor}}) \textendash{} AutoFeatureExtractor from pretrained model

\item {} 
\sphinxAtStartPar
\sphinxstyleliteralstrong{\sphinxupquote{max\_duration}} (\sphinxstyleliteralemphasis{\sphinxupquote{int}}) \textendash{} maximum seconds recording

\end{itemize}

\sphinxlineitem{Returns}
\sphinxAtStartPar
the preprocessed data as AutoFeatureExtractor

\sphinxlineitem{Return type}
\sphinxAtStartPar
AutoFeatureExtractor

\end{description}\end{quote}

\end{fulllineitems}

\index{reads() (in module auxilary\_functions)@\spxentry{reads()}\spxextra{in module auxilary\_functions}}

\begin{fulllineitems}
\phantomsection\label{\detokenize{auxilary_functions:auxilary_functions.reads}}
\pysigstartsignatures
\pysiglinewithargsret{\sphinxcode{\sphinxupquote{auxilary\_functions.}}\sphinxbfcode{\sphinxupquote{reads}}}{\sphinxparam{\DUrole{n}{x}}}{}
\pysigstopsignatures
\sphinxAtStartPar
Function to extract the number of Reads from a string
\begin{quote}\begin{description}
\sphinxlineitem{Parameters}
\sphinxAtStartPar
\sphinxstyleliteralstrong{\sphinxupquote{x}} (\sphinxstyleliteralemphasis{\sphinxupquote{str}}) \textendash{} string which has a key word \sphinxhyphen{} Reads to extarct the info from

\sphinxlineitem{Returns}
\sphinxAtStartPar
Number of Reads

\sphinxlineitem{Return type}
\sphinxAtStartPar
int

\end{description}\end{quote}

\end{fulllineitems}

\index{recommendations() (in module auxilary\_functions)@\spxentry{recommendations()}\spxextra{in module auxilary\_functions}}

\begin{fulllineitems}
\phantomsection\label{\detokenize{auxilary_functions:auxilary_functions.recommendations}}
\pysigstartsignatures
\pysiglinewithargsret{\sphinxcode{\sphinxupquote{auxilary\_functions.}}\sphinxbfcode{\sphinxupquote{recommendations}}}{\sphinxparam{\DUrole{n}{x}}}{}
\pysigstopsignatures
\sphinxAtStartPar
Function to extract the number of Recommendations from a string
\begin{quote}\begin{description}
\sphinxlineitem{Parameters}
\sphinxAtStartPar
\sphinxstyleliteralstrong{\sphinxupquote{x}} (\sphinxstyleliteralemphasis{\sphinxupquote{str}}) \textendash{} string which has a key word \sphinxhyphen{} Recommendations to extarct the info from

\sphinxlineitem{Returns}
\sphinxAtStartPar
Number of Recommendations

\sphinxlineitem{Return type}
\sphinxAtStartPar
int

\end{description}\end{quote}

\end{fulllineitems}

\index{split\_list() (in module auxilary\_functions)@\spxentry{split\_list()}\spxextra{in module auxilary\_functions}}

\begin{fulllineitems}
\phantomsection\label{\detokenize{auxilary_functions:auxilary_functions.split_list}}
\pysigstartsignatures
\pysiglinewithargsret{\sphinxcode{\sphinxupquote{auxilary\_functions.}}\sphinxbfcode{\sphinxupquote{split\_list}}}{\sphinxparam{\DUrole{n}{row}}\sphinxparamcomma \sphinxparam{\DUrole{n}{column\_name}}}{}
\pysigstopsignatures
\sphinxAtStartPar
The goal of this function is to split a column which consists of a list into several columns.
\begin{quote}\begin{description}
\sphinxlineitem{Parameters}\begin{itemize}
\item {} 
\sphinxAtStartPar
\sphinxstyleliteralstrong{\sphinxupquote{row}} (\sphinxstyleliteralemphasis{\sphinxupquote{pd.Series}}) \textendash{} row from a data frame

\item {} 
\sphinxAtStartPar
\sphinxstyleliteralstrong{\sphinxupquote{column\_name}} (\sphinxstyleliteralemphasis{\sphinxupquote{str}}) \textendash{} name of the column with the list

\end{itemize}

\sphinxlineitem{Returns}
\sphinxAtStartPar
pd.Series with the unpacked list

\sphinxlineitem{Return type}
\sphinxAtStartPar
pd.Series

\end{description}\end{quote}

\end{fulllineitems}

\index{start\_page() (in module auxilary\_functions)@\spxentry{start\_page()}\spxextra{in module auxilary\_functions}}

\begin{fulllineitems}
\phantomsection\label{\detokenize{auxilary_functions:auxilary_functions.start_page}}
\pysigstartsignatures
\pysiglinewithargsret{\sphinxcode{\sphinxupquote{auxilary\_functions.}}\sphinxbfcode{\sphinxupquote{start\_page}}}{\sphinxparam{\DUrole{n}{pdf}}}{}
\pysigstopsignatures
\sphinxAtStartPar
Genrates pdf file
\begin{quote}\begin{description}
\sphinxlineitem{Parameters}
\sphinxAtStartPar
\sphinxstyleliteralstrong{\sphinxupquote{pdf}} (\sphinxstyleliteralemphasis{\sphinxupquote{FPDF}}) \textendash{} FPDF instance

\sphinxlineitem{Returns}
\sphinxAtStartPar
pdf file

\end{description}\end{quote}

\end{fulllineitems}



\chapter{Indices and tables}
\label{\detokenize{index:indices-and-tables}}\begin{itemize}
\item {} 
\sphinxAtStartPar
\DUrole{xref,std,std-ref}{genindex}

\item {} 
\sphinxAtStartPar
\DUrole{xref,std,std-ref}{modindex}

\item {} 
\sphinxAtStartPar
\DUrole{xref,std,std-ref}{search}

\end{itemize}


\renewcommand{\indexname}{Python Module Index}
\begin{sphinxtheindex}
\let\bigletter\sphinxstyleindexlettergroup
\bigletter{a}
\item\relax\sphinxstyleindexentry{auxilary\_functions}\sphinxstyleindexpageref{auxilary_functions:\detokenize{module-auxilary_functions}}
\indexspace
\bigletter{b}
\item\relax\sphinxstyleindexentry{BeeClassification}\sphinxstyleindexpageref{BeeClassification:\detokenize{module-BeeClassification}}
\item\relax\sphinxstyleindexentry{BeeData}\sphinxstyleindexpageref{BeeData:\detokenize{module-BeeData}}
\item\relax\sphinxstyleindexentry{BeeLitReview}\sphinxstyleindexpageref{BeeLitReview:\detokenize{module-BeeLitReview}}
\end{sphinxtheindex}

\renewcommand{\indexname}{Index}
\printindex
\end{document}